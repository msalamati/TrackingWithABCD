\section{Discussion and Future Work}
We have presented a method for synthesizing provably correct decentralized controllers for multi-agent systems with global reach-avoid specifications. 
Our approach is hierarchical: At the top level, we obtain a high-level joint plan by solving a centralized synthesis problem on a simplified model of the product system, and at the bottom level we synthesize local controllers for robustly tracking the high-level plan for each individual agent.
In this section, we discuss different features of our work, remaining challenges, and the future work.

\smallskip
\noindent\textbf{Scalability of the proposed method.}\
We compute the reference trajectory for the product system using ALTRO. Our experiments show that ALTRO generates reference trajectories for high-dimension systems quite fast. On the other hand, for tracking controllers, our method synthesizes controllers in a decentralized manner and thus scalability is not affected by increasing the number of agents. We use SCOTS for computing tracking controllers because the algorithm can be effectively parallelized \cite{KhaledZ19pfaces}. While \tool uses ALTRO and SCOTS for solving the given reach-avoid problem, we emphasize that one could replace them with any other off-the-shelf method that can perform similar tasks.

\smallskip
\noindent\textbf{Comparison to other tools that are able to solve reach-avoid specifications.}\
Our experiments show that our method is efficient in solving reach-avoid tasks for multi-agent systems. However, there are other tools which claim they can solve similar tasks. Table~\ref{tab:tools} lists main features for some of such active tools \MZ{that are public availble and we aware of them}.
\begin{table*}[t]
	\large
	\centering
	\caption{Features of different tools.}\label{tab:tools}
	\renewcommand{\arraystretch}{1.2}
	\setlength{\tabcolsep}{0.7em} % for the horizontal padding
	%\resizebox{1\columnwidth}{!}{
	\begin{tabular}{l|>{\centering}m{18mm}|>{\centering}m{18mm}|>{\centering}m{18mm}|>{\centering\arraybackslash}m{18mm}}
		\toprule
		Tool &
		Non-linear Dynamics & Formal Guarantee & Multi-agent & Decentralized controllers\\
		\hline
		\tool  & \ding{51} &\ding{51} & \ding{51} &  \ding{51}\\
		\hline
		FastTrack\cite{herbert2017fastrack}& \ding{51} & \ding{51} &   &   \\
		\hline
		RealSyn\cite{fan2018controller}& &\ding{51} &  &  \\
		\hline
		Factest\cite{fan2020fast}& \ding{51} &  &  &  \\
		\hline
		Model mismatch(SOS)\cite{singh2018robust}  & \ding{51} &  &  &  \\
		\hline
		RTD\cite{kousik2020bridging} & \ding{51} & \ding{51} &  &  \\
		\hline
		Fly by logic\cite{Pant2018multiquad}  & \ding{51} & & \ding{51} & \ding{51} \\
		\hline
		Distributed team lift \cite{jackson2020scalable} & \ding{51} &  & \ding{51} &  \\
		\bottomrule
	\end{tabular}
\end{table*}

\smallskip
\noindent\textbf{Extension to richer classes of specifications.}\
Our proposed method is specific to reach-avoid specifications. However, application of our method can be extended to the problem that can be broken into a sequence of reach-avoid tasks. To that end, one can use high-level languages (e.g., \cite{Majumdar2020,Ghosh2020}) for specifying complex reach-avoid sequences and invoke \tool to solve each individual reach-avoid task.

\smallskip
\noindent\textbf{Choice of parameters.}\
Our algorithm takes as input the robustness margin $\varepsilon$ and the abstraction parameters $\eta_x$ and $\eta_u$.
(The safety margin $\delta$ is also a parameter, but that is considered to be a part of the control problem.)
The larger the value of $\varepsilon$, more difficult it is to synthesize a nominal controller for the product system using ALTRO.
On the other hand, the smaller the value of $\varepsilon$, the more difficult it is to synthesize a set of decentralized controllers using SCOTS.
Also, there is a trade-off between computation time and ease of synthesis when it comes to choosing the parameters $\eta_x$ and $\eta_u$ for SCOTS: Larger values of these parameters will result in faster computation but harder synthesis problem.
Thus there needs to be a good balance when we choose the parameters for our tool.
Indeed, it is not always easy to come up with a good set of parameters.
Our current approach is to add an outer loop around our tool to search of suitable parameters until the decentralized synthesis is successful (see Fig.~\ref{fig:overall}).
In future, we plan to investigate if the parameters can be systematically discovered by separately analyzing the dynamics of the systems.

%\smallskip
%\noindent\textbf{Conservativeness of our method.}\
%Our method is no more conservative than the standard ABCD approach used in SCOTS:
%Indeed, assuming availability of unbounded resources, if SCOTS is able to find a central controller for the global synthesis problem for a given set of parameters $\varepsilon$, $\eta_x$, and $\eta_u$, then our method will also find a set of decentralized controllers (by consuming much less resource than the global synthesis approach).
%\KM{@Mehrdad: is this true? Is it not the case that ALTRO can occassionally fail to give a nominal controller even if the central SCOTS-based approach would have solution, if we would ignore the memory and time limitations?}


%\section{conclusion}
%\MS{We have presented a decentralized controller synthesis scheme for multi-agent systems with global reach-avoid specifications. Our experiments demonstrate effectiveness of our approach for solving reach-avoid problems that may require formation to be preserved. Note that one can connect our work to the high-level languages for robotic applications by formulating more complicated tasks as a sequence of reach-avoid problems together with correct synchronization skleton. It remains to choose the right values of parameters for robustness parameter $\varepsilon$ and discretization parameters $\eta_x$ and $\eta_u$.}
