\section{discussion}
\MS{We have presented a decentralized controller synthesis scheme for multi-agent systems with global reach-avoid specifications. In this section, we highlight the features of our work, remaining challenges and the future work.

\smallskip
\noindent\textbf{Scalability of the proposed method.}\
We compute the reference trajectory for the product system using ALTRO. Our experiments show that ALTRO generates reference trajectories for high-dimension systems quite fast. On the other hand, for tracking controllers, our method synthesizes controllers in a decentralized manner and thus scalability is not affected by increasing the number of agents. We use SCOTS for computing tracking controllers because the algorithm can be effectively parallelized \cite{KhaledZ19pfaces}. While our tool uses ALTRO and SCOTS for solving the given reach-avoid problem, we emphasize that one could replace them with any other off-the-shelf method that can perform similar tasks.

\smallskip
\noindent\textbf{Comparison to other tools that are able to solve reach-avoid specifications.}\
Our experiments show that our method is efficient in solving reach-avoid tasks for multi-agent systems. However, there are other tools which claim they can solve similar tasks. Table~\ref{tab:tools} lists main features for some of such active tools.
\begin{table*}[t]
	\large
	\centering
	\caption{Features of different tools.}\label{tab:tools}
	\renewcommand{\arraystretch}{1.2}
	\setlength{\tabcolsep}{0.7em} % for the horizontal padding
	%\resizebox{1\columnwidth}{!}{
	\begin{tabular}{l|c|c|c}
		\toprule
		Tool&Method&
		Dynamics&Type of disturbance\\
		\midrule
		Ours(??)&decentralized&non-linear&bounded additive\\
		\midrule
		FastTrack&centralized&non-linear&-\\
		\midrule
		Factest&??&??&??\\
		\midrule
		RTD&??&??&??\\
		\bottomrule
	\end{tabular}
\end{table*}

\smallskip
\noindent\textbf{Extension to richer classes of specification.}\
Our proposed method is specific to reach-avoid specifications. However, application of our method can be extended to the problem that can be broken into a sequence of reach-avoid tasks. To that end, one can use high-level languages (e.g., \cite{Majumdar2020,Ghosh2020}) for specifying complex reach-avoid sequences and invoke our tool to solve each individual reach-avoid task.

\smallskip
\noindent\textbf{Choice of parameters.}\

\smallskip
\noindent\textbf{Conservativeness of our method.}\
}

%\section{conclusion}
%\MS{We have presented a decentralized controller synthesis scheme for multi-agent systems with global reach-avoid specifications. Our experiments demonstrate effectiveness of our approach for solving reach-avoid problems that may require formation to be preserved. Note that one can connect our work to the high-level languages for robotic applications by formulating more complicated tasks as a sequence of reach-avoid problems together with correct synchronization skleton. It remains to choose the right values of parameters for robustness parameter $\varepsilon$ and discretization parameters $\eta_x$ and $\eta_u$.}