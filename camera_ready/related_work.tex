% !TEX root = main.tex

%\smallskip
\noindent\textbf{Related Work}
%
The field of multi-agent planning is too large for a comprehensive survey; we point to the text books
\cite{LaValle2006,LaValle1998planning,choset2005principles,russel2010AIplanning} for an introduction.
We categorize closely related work into 
(1) those combining planning and tracking controller synthesis, 
(2) those addressing formal multi-agent controller synthesis, and 
(3) those combining (1) and (2). We provide a survey of these categories. 

	\smallskip\noindent\emph{Combining planning and tracking.} 
Techniques combining high-level planning and low-level tracking are a staple of classical planning and control. 
More recently, several techniques consider the problem of formal guarantees for such planners.
Existing works differ in the dynamics that they can handle (e.g., linear or nonlinear),
considered class of specifications, including disturbances, and scalability. 
A common approach is to perform the high-level planning over a lower dimensional model and then use
Sum-of-Squares programming (SOS), Hamilton Jacobi (HJ), or satisfiability modulo convex programming (SMC) 
to obtain a low-level controller ensuring a bounded error between the 
two models \cite{herbert2017fastrack,meyer2019,singh2018robust,Nilsson:2018}. 

%\KM{Nothing has been stated about \cite{DBLP:journals/corr/abs-1911-09773,sun2019}.} 
In contrast to \cite{herbert2017fastrack,singh2018robust}, our method does not require finding 
a linear mapping between the low and high dimensional models. 
Meyer et al.~\cite{meyer2019} considered reach-avoid problems for perturbed non-linear control affine systems. 
They create a lower dimensional model and use SOS programming to compute a controller ensuring a bounded error between the two models. 
Then, they use ABCD to compute a controller for the low-order model while taking the error into account. 
While their method can provide guarantee against worst-case disturbances, it is not clear if SOS always scales to the higher dimensions.

Nilsson et al.~\cite{Nilsson:2018} provide a method that decomposes the state space into a lower-order planning space, 
and a higher-order internal dynamics space, so that fast planning and accurate tracking can be achieved using a set of control barrier functions computed based on SOS. Despite providing guarantees for the worst-case bounded disturbances, their method is not capable of solving reach-avoid tasks which involve dynamic obstacles as in the multi-agent case.
%that governs the motion in planning space. 
%The planning is performed fast since it is restricted to the lower-order space. To guarantee following of the planned trajectory in the lower-order space, control barrier functions are computed using SOS method. This method, however, cannot deal with dynamic obstacles as is needed in tackling collision avoidance.}
While we have chosen SCOTS since the underlying algorithm can be effectively parallelized \cite{KhaledZ19pfaces}, in principle, we could
also use SOS, HJ, or SMC approaches. 
%The existing schemes that solve the guaranteed motion planning problems in a centralized manner (e.g. \cite{sun2019}) cannot scale well.

Other works only consider special classes of models such as linear \cite{fan2018controller,wongpiromsarn2012receding,Rodionova2020LearningtoFlyLC}, 
disturbance-free \cite{tedrake2010lqr,fan2020fast,Srinivasan2018}, or finite transition systems \cite{Yang2017milp}. 
In contrast, our method supports arbitrary 
nonlinear dynamics and provides a guarantee 
against worst-case bounded disturbances.

	\smallskip\noindent\emph{Formal multi-agent synthesis.} 
Chen et al.\ \cite{Chen2018cbf} provide a method, using control barrier functions, that requires some form of inter-robot 
communication and does not consider external disturbances.
Sahin et al.\ \cite{Shahin2017cltl} propose a method that requires the group of robots to be homogeneous.
There are methods which do not consider external disturbances and do not provide formal guarantees \cite{jackson2020scalable}.

\begin{table}[t]
	%\large
	\centering
	\caption{Features of the publicly available tools compared to \tool. Note that some of these tools can handle richer classes of specifications, compared to the reach-avoid problem handled by \tool.}\label{tab:tools}
	\renewcommand{\arraystretch}{1.2}
	\setlength{\tabcolsep}{0.7em} % for the horizontal padding
	%\resizebox{1\columnwidth}{!}{
	\begin{tabular}{l 
							>{\centering}m{5mm}
							>{\centering}m{5mm} 
							>{\centering}m{5mm}	
							>{\centering\arraybackslash}m{5mm}}
		\toprule
		Tool name &
		\rotatebox{90}{Non-linear Dynamics} & \rotatebox{90}{Formal Guarantee} & \rotatebox{90}{Multi-agent} & \rotatebox{90}{Decentralized Controllers}\\
		\hline
		\rowcolor{black!20} \tool  & \ding{51} &\ding{51} & \ding{51} &  \ding{51}\\
		\hline
		SCOTS \cite{Rungger2016scots}	&	\ding{51}	&	\ding{51}		&		&	\\
		\hline
		ALTRO \cite{howell2019altro}		&	\ding{51}	&	&	&\\
		\hline
		FastTrack \cite{herbert2017fastrack}& \ding{51} & \ding{51} &   &   \\
		\hline
		RealSyn \cite{fan2018controller}& &\ding{51} &  &  \\
		\hline
		Factest \cite{fan2020fast}& \ding{51} &  &  &  \\
		\hline
		Model mismatch (SOS) \cite{singh2018robust}  & \ding{51} &  &  &  \\
		\hline
		RTD \cite{kousik2020bridging} & \ding{51} & \ding{51} &  &  \\
		\hline
		Fly-by-Logic \cite{Pant2018multiquad}  & \ding{51} & & \ding{51} & \ding{51} \\
		\hline
		Distributed team lift \cite{jackson2020scalable} & \ding{51} &  & \ding{51} &  \\
		\bottomrule
	\end{tabular}
\end{table}
	\smallskip\noindent\emph{Combinations.} 
Alonso-Mora et al.\ \cite{alonso2019distributed} provide a method for formation control of a group of communicating homogeneous robots.
They first synthesize a nominal controller using a fast randomized geometric planning method, namely RRT, and 
then use optimal control to track the obtained nominal solution.
Unlike us, they neither consider external disturbances nor provide formal guarantees. 
Pant et al.~\cite{Pant2018multiquad} have studied multi-quadrotor missions with signal temporal logic (STL) specifications. 
They find the reference trajectory by maximizing robustness of the STL specification, and then synthesize tracking controllers. 
Their method can only handle specifications with bounded horizon and does not provide any guarantee against disturbances.

Xiao et al.\ \cite{xiao2019merging} propose a method for synthesis of distributed controllers for a 
set of autonomous vehicles in a lane merging situation. 
They consider only linear systems as vehicle models, use global optimal control to find a nominal controller, 
and employ local control barrier functions with safety constraints.
Their designed controllers is not provably safe in the presence of disturbance and can occasionally violate the safety constraints.
 %
Nikou et al.\ \cite{Nikou2019} have studied the problem of robust navigation for multi-agent systems based on nominal reference 
trajectory and pre-computed feedback controllers. 
Their approach requires sensing capabilities of the agents to avoid collision. 
In contrast, our method does not requires any sensing capabilities of the agents.
%
%They consider additive bounded modelling disturbance. 
%They use optimal control to compute a  and employ a  to keep the trajectories within a bounded distance from the reference trajectory.
Sun et al.\ \cite{sun2019} have studied motion planning of multi-agent systems with linear temporal logic (LTL) specifications, under the presence of disturbances and denial of service attacks.
Their approach uses SMC programming to compute a feasible nominal trajectory and employs feedback controllers to gain robustness.
%First, they compute a feedback controller that forces the system's trajectories to remain within a robust control invariant set under the worst-case disturbance scenario. 
%For piecewise-affine systems, there exist effective algorithms for computing the feedback controller and the corresponding invariant set. 
%Next, they use  together with an open-loop controller, and by adding up the nominal open-loop control input and the feedback control law they construct the final control input.
Despite being able to provide guarantees against disturbances, their implementation is centralized, thus 
the required time increases significantly for high-dimensional reach-avoid specifications.
%The main drawback of their method is central implementation, since tackling reach-avoid problems in higher dimensions takes significantly longer time.
%However, they assume that agents are able to communicate while moving; and their target is to satisfy input-to-state stability. In fact, their method is not able to tackle collision avoidance for static/dynamic obstacles.}

There are other works that use a pre-defined motion primitive library to perform planning for multi-robot 
systems \cite{saha2016implan,BanusicMPSZ19pgcd,Gavran2017antlab,desai2017drona}. 
In contrast, our method deals with the dynamical model directly.

Our construction can also be seen as an \emph{assume-guarantee} technique that decomposes the global problem based on nominal trajectory tubes. Similar decompositions have been studied in the discrete case \cite{alur2015pattern,majumdarassume}.
The closest related work that matches our level of generality is the work by Bansal et al.\ \cite{bansal2017safe}.
However, they assume that each robot has its own reach-avoid specification while avoiding collision with the other robots.
In contrast, we allow global reach-avoid specifications, which subsume their class of specifications.
In fact, there are control problems that can be easily handled by our approach and cannot be encoded in their setting. An example is robots maintaining a formation while fulfilling their tasks \cite{alonso2019distributed}.

%Our experiments show that \tool is efficient in solving reach-avoid tasks for multi-agent systems. 
%However, there are other tools which claim they can solve similar tasks. 
A subset of approaches listed above have available implementations.
In Table~\ref{tab:tools}, we summarize the main features of the publicly available tools.
We highlight that our tool \tool is the only one that fulfills all the criteria.


	%The hierarchical approach, combining high-level planning and low-level tracking, is not new.
	%Yet most of the existing works use the hierarchical approach for controller synthesis for a single robot \cite{herbert2017fastrack, fan2020fast, fan2018controller, wongpiromsarn2012receding, tedrake2010lqr, singh2018robust, DBLP:journals/tac/MeyerD19, DBLP:journals/corr/abs-1911-09773, Yang2017milp}.
	%In contrast, we consider a hierarchical control for decentralized control of a group of robots.
%	For example, there are methods which can generate nominal controlled trajectories by taking into account the maximum tracking error, assuming that a tracking controller is already in place \cite{fan2020fast, fan2018controller}.
%	On the other hand, there are methods which compute a tracking controller which minimizes the tracking error for any given nominal trajectory:
%	Some of them do not consider external disturbances \cite{tedrake2010lqr, singh2018robust}, some of them only consider linear dynamics \cite{wongpiromsarn2012receding}, and all of them consider only a single monolithic system structure \cite{herbert2017fastrack}.

% Many papers address the problem of multi-robot controller synthesis without a hierarchical approach.


% In contrast, our method can synthesize formally verified controllers for heterogeneous group of robots, subjected 
% to external disturbances, and lacking any form of communications. 
% Hierarchical controller synthesis for decentralized control of multi-robot systems is relatively rare.

% In contrast, our method provides guarantee against worst-case system noise at all time, and in addition can support nonlinear systems. 
 

%\begin{enumerate}
%	\item FastTrack: Guaranteed tracking of nonlinear trajectories for nonlinear systems under disturbances \cite{herbert2017fastrack}. \MS{They only account for the error between the planning and tracking models and not for additive disturbance applied to the tracking model. They also rely on possibility of relating the two afformentioned models by a linear mapping (which is not the case for the general class of systems). Finally, while their method scales up better than (pure) SCOTS, it does not scale up to dimensions above ten.}
%%	\item Using the tracking error in the planning phase to make sure that the controller will be safe \cite{fan2020fast}. Relies on an expert user to already provide the low-level tracking controller, whereas we synthesize the tracking controller. So they address the inverse problem in some sense.
%%	\item Another paper of the same flavor as the previous one \cite{fan2018controller}. Here they restrict themselves to discrete time perturbed linear systems.
%%	\item Sum-of-squares based tracking methods \cite{tedrake2010lqr, singh2018robust}. Do not consider external disturbances.
%	\item Other hierarchical methods using planning+tracking like approach for formal methods-based synthesis for single systems \cite{wongpiromsarn2012receding, DBLP:journals/tac/MeyerD19, DBLP:journals/corr/abs-1911-09773}.
%	
%	\item A robust scheme for controlling linear dynamical systems with additive bounded noise and (general) LTL specification \cite{Yang2017milp}. They use mixed integer linear programming (MILP) to design a nominal controller satisfying a given LTL specification in the absence of noise and then design a regulating controller to account for the noise term.
%\end{enumerate}

%\smallskip
%\noindent\textbf{Ones which address multi-robot control problems.}\
%
%\KM{The plan is to collect all the papers first in the following list. After that, I will make a story out of them.}
%Here are the papers that we know of:
%\begin{enumerate}
%	\item Safe control for multi-vehicle system under presence of disturbances \cite{bansal2017safe}. In contrast to our method, they assume a fixed priority order of the vehicles: The higher priority vehicles make independent plans, which are treated as (time-varying) obstacles for the lower priority vehicles. 
%	\new{One way our method is probably more powerful is if we consider controller synthesis for mechanically coupled systems with joint goal, or even formation control problem for mechanically decoupled systems. For these class of problems, I don't think their method will directly apply, whereas ours should work fine.}
%	
%	\item A method to control a robot swarm (modelled as planar robots with potentially different speed bounds) such that a given specification (expressable in a fragment of LTL) is satisfied \cite{Chen2018cbf}. Some extent of communication between robots is required. The collision avoidance is guaranteed via designing control barrier functions for each pair of robots. No source of disturbance is considered in the model and the proposed solution does not work for agents modeled not as planar robots.
%	\item Controlling (homogeneous) robot swarms (modelled as finite state transition systems) for satisfying specifications expressed in a temporal logic for reasoning about the collective behavior of multiple agents (cLTL) \cite{Shahin2017cltl}. Their method can potantially account for bounded noise and non-linearities, but not heterogeneity.
%	\item Controlling multi-robot (it can be heterogeneous) using ALTRO \cite{jackson2020scalable}. very similar to our work without formal guarantee but works in real-time. they are making centralized system by combining dynamics of several quads. 
%	\item The paper that we got merging example from that\cite{xiao2019merging}. They are using optimal control + control barrier function. they are can handle disturbance and in addition they are optimizing time and fuel consummation and more constraints but they are considering linear model.
%	\item Hierarchical control for interconnected discrete event systems \cite{wong1996hierarchical,schmidt2008nonblocking}.
% \end{enumerate}
