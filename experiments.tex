%! TEX ROOT = ./main.tex

\section{Experimental Evaluation}\label{sec:experiments}
\MS{List the specifications of the cluster machine/laptop with which we have run our experiments.}
\subsection{Inverted Pendulum}\label{subsec:invpend}
As a simple example, consider the standard problem of designing a controller for a frictionless inverted pendulum.
The goal of the controller is to bring the pendulum at the vertical position while satisfying hard constraints on the state variables and control inputs.
A model of the system can be constructed using physical principles and is described with the following state space representation
\begin{align*}\label{eq:inv_pend_ss}
\dot\theta(t)&=\omega(t)+w_1(t)\nonumber\\
\dot\omega(t)&=\frac{g}{L}sin(\theta(t))+u(t)+w_2(t)
\end{align*}
where $\theta(t)$, $\omega(t)$, and $u(t)$ denote respectively the angular position, angular speed, 
and the input torque at time $t$. The disturbance $w=[w_1,w2]^T\in\mathbb R^2$, bounded with a polyhedral set $w(t)\in \mathcal W$, captures the modeling errors. 
The parameter $g=9.81 [m/s^2]$ is the gravitational acceleration, $m$ is the ball mass
and $L$ is the length of the bar. 

Starting from an initial state $[\theta_0,\omega_0]=[\pi,0]$, the control goal is to converge to the equilibrium point 
$\theta=0, \omega=0$. To this end, we first used Altro (\MS{\cite{??}}) to generate a (nominal) trajectory along which the state vector can travel from $[\pi,0]$ to $[0,0]$. Setting the time step $T_s=0.05$, $L=9.81$ and \MS{$N=??$}, generating the nominal trajectory takes around \MS{??} seconds. Next, we use Scots in order to synthesize a controller such that the nominal trajectory is not violated beyond the specified tube around it, under different levels of disturbance. Setting \MS{$\eta_s=[?,?]$} and \MS{$\eta_i=[?]$}, Figure~\ref{fig:inv_pend} demonstrates variations of disturbance limit for which Scots is able to find a controller for different tube width. It can be observed that by increasing the tube width, Scots is able to synthesize a controller for larger disturbance levels. Figure~\ref{fig:invpend_traj} demonstartes trajectories of the system  \eqref{eq:inv_pend_ssq} governed by controllers synthesized by Altro and Scots under appliance of constant disturbance vector \MS{$[??,??]$}. As expected, the controller designed by Altro is not able to tackle disturbance, while using the controller generated by Scots, the trajectory remains within the desired bound.
%Additionally, we require the state constraints $\theta_k\in[-\pi,\pi]$ and $\omega_k\in[-\pi/8,\pi/8]$ to hold at all time instances.
%\begin{figure}\label{fig:inv_pend}
%	\includegraphics[]{}
%	\caption{Variations of disturbance limit for which Scots is able to find a controller with tube size}
%\end{figure}
%\begin{figure}\label{fig:invpend_traj}
%	\includegraphics[]{}
%	\caption{}
%\end{figure}
\subsection{Ship Docking}\label{subsec:6dship}
Next, we consider the problem of ship docking: a ship that has been waiting for finding an appropriate slot, starts moving toward the dock from a certain distance. The control goal is hence to bring the ship from its start point into the dock such that the ship's speed near the dock is very small. To represent the ship's model in state space, we partition the state vector as $x=[\eta,\nu]$, where $\eta=[N,E,\psi]^T$ are the South-North and West-East positions and heading of the ship, $\nu = [u ,v ,r]^T$ are the surge and sway velocities, and yaw rate of the ship. Moreover the disturbance vector is partitioned as $w=[w_c,w_{wind}]^T$ where $w_c\in\mathbb R^3$ is the disturbance due to the current velocities and $w_{wind}\in\mathbb R^3$ is the disturbance corresponding to wind forces. The state space equations of the ship model are
\begin{align*}
&\dot{\eta}=R(\psi)\nu+w_c \\
&M\dot{\nu}+C(\nu)\nu+D\nu=u+R(\psi)^{\top}w_{wind},
\end{align*}
where $u\in\mathbb R^3$ denotes the control input vector and $R=R(\psi)=\begin{bmatrix}
cos(\psi) &-sin(\psi) &0\\
sin(\psi) & cos(\psi) & 0\\
0 & 0 & 1
\end{bmatrix}$ is the rotation matrix. Further, the inertia matrix $M=\begin{bmatrix}
87.4 & 0 & 0 \\
0 & 98.3 & 2.48 \\
0 & 2.48 & 22.2
\end{bmatrix}$, the damping matrix $D=\begin{bmatrix}
6.58 & 0 & 0 \\
0 & 37.7 & 2.66 \\
0 & 2.66 & 19.3
\end{bmatrix}$ and Coriolis matrix $C(v)=\nu(1)\begin{bmatrix}
0 & 0 & 0 \\
0 & 0 & 98.3 \\
0 & 0 & 2.48
\end{bmatrix}$ are chosen for a $1:30$ scale model of a platform supply vessel.

Starting from an initial state $[0,0,0,0,0,0]$, the control goal is to converge to the point 
$[1,1,0,0,0,0]$. To this end, we first used Altro to generate a (nominal) trajectory. Setting the time step $T_s=3$, and \MS{$N=??$}, generating the nominal trajectory takes around \MS{??} seconds. Next, we use Scots in order to synthesize a controller such that the nominal trajectory is not violated beyond the specified tube around it, under different levels of disturbance. Setting \MS{$\eta_s=[?,?,?,?,?,?]$} and \MS{$\eta_i=[?,?,?]$}, Figure~\ref{fig:inv_pend} demonstrates variations of disturbance limit for which Scots is able to find a controller for different tube width. %It can be observed that by increasing the tube width, Scots is able to synthesize a controller for larger disturbance levels. Figure~\ref{fig:invpend_traj} demonstartes trajectories of the system  \eqref{eq:inv_pend_ssq} governed by controllers synthesized by Altro and Scots under appliance of constant disturbance vector \MS{$[??,??]$}. As expected, the controller designed by Altro is not able to tackle disturbance, while using the controller generated by Scots, the trajectory remains within the desired bound.

%\begin{figure}\label{fig:inv_pend}
%	\includegraphics[]{}
%	\caption{Variations of disturbance limit for which Scots is able to find a controller with tube size}
%\end{figure}
%\begin{figure}\label{fig:invpend_traj}
%	\includegraphics[]{}
%	\caption{}
%\end{figure}




