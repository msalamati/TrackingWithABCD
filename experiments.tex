%!TEX ROOT = ./main.tex

\section{Experimental Evaluation}\label{sec:experiments}
%\MS{List the specifications of the cluster machine/laptop with which we have run our experiments.}
We evaluate a prototype implementation of ALg.~\ref{alg:abcd-for-tracking}
on two examples. 
The design of nominal controller has been performed on a laptop with core i7-4510u CPU at 3.10GHz, with 8GB of
RAM.
The formal controller synthesis has been performed
on a cluster with 4 Intel Xeon E7-8857 v2 CPUs (48 cores in total) at 3GHz, with 1.5TB of
RAM.

%\textcolor{red}{
\subsection{Multi-car path planning}\label{sec:MultiAgent}

In this example, we consider a planning scenario for ten robots modeled identically as tuple $\Sigma=(X,U,W,f^u)$. The set of state variables $X$ consists of $x_1$, $x_2$ and $x_3$ which denote position in $2-D$ plane and movement angle, respectively. The set of input variables includes $u_1$ and $u_2$ which represent linear and angular speeds of each robot. Finally,
\begin{equation}\label{eq:unicycle_ss}
	f^{u}(x(t),u(t))=
	\begin{bmatrix}
		\dot{x_1}\\
		\dot{x_2}\\
		\dot{x_3}
	\end{bmatrix}=
	\begin{bmatrix}
		u_1cos(x_3)\\
		u_1sin(x_3)\\
		u_2
	\end{bmatrix}.
\end{equation}
%In the above equation, $u_1(t)$ and $u_2(t)$ denote linear and angular speeds at time $t$, respectively. Further, $x_1(t)$, $x_2(t)$ specify position in $x-y$ plane and $x_3$ denotes the angle at time $t$. 

A set of initial states ($Init$) and target sets ($Goal$) for each robot is specified and depicted in Figure~\ref{fig:MA} with red circles and green triangles, respectively. A physical obstacle is added to the problem over the $2-D$ plane, depicted in Figure~\ref{fig:MA} with a yellow circle. The thresholds for defining the static obstacle are chosen as $\delta_{col}=1.5$ and $\delta_{obs}=2$. The control objective is to synthesize a controller for each robot so that in presence of additive bounded disturbance, beginning from the set $Init^i$, a corresponding target set $Goal^i$ is reached within a finite horizon, while avoiding the static obstacle $Obs$ at every time point. 

According to our proposed scheme, we first use ALTRO to design a centralized nominal controller for the product state and input space that satisfies specification in Equation~\eqref{eq:spec} in the absence of disturbance ($W=\{0\}$). We choose the horizon length $K=200$ and sampling time $\tau=0.05$. ALTRO computes a valid open-loop controller in $138$ seconds for the product space with $30$ state and $20$ input variables. The open-loop trajectories generated by ALTRO ($(x_{0_\nom}^i,\ldots,x_{K_\nom}^i)$) are depicted in solid in Figure~\ref{fig:MA}. Note that the guarantee that open-loop controller gives is only valid in the absence of disturbance. In Figure~\ref{fig:MA}, dotted trajectories correspond to the first and the second robot's paths in 2-D plane when constant additive disturbance $w=\{\begin{bmatrix}0.03,0.03,0.03\end{bmatrix}\}$ is being applied throughout the whole horizon. It can be noticed that they collide with each other before reaching their corresponding target sets. Next, we show how to synthesize local controllers that guarantee satisfaction of specifications in the presence of disturbance. \MS{1) Use dashed/dotted for trajectories with added disturbance. 2) use $x_1$ and $x_2$ instead of $x$ and $y$ for different axes 3) don't draw the outer circle.}

We augment each open-loop trajectory with time and construct a tube around it with the width vector $\varepsilon=\begin{bmatrix}0.16&0.16&0.16&0\end{bmatrix}$. Note that $\delta_{col} > 2\sqrt{\varepsilon_x^2+\varepsilon_y^2}$. Provided that we have computed the set for each of the robots, we are ready to use SCOTS in order to synthesize (local) closed-loop controllers that guarantee reachability and obstacle avoidance in the presence of bounded disturbance. We consider state and input spaces to be $Domain=[-1:11]\times[-1:11]\times[-\pi,\pi]\times[0,1]$ and
$U=[-3,3]\times[-3,3]$, respectively. Choosing state and input partition sizes $\eta_{\widetilde{X}}=[0.04,0.04,0.04,0.05]$ and
$\eta_{U}=[0.6,0.6]$ results in $\hat X$ and $\hat U$ with $7\times 10^7$ and $121$ points. \MS{What are the numbers after restricting the abstraction into tube's inner space?}The bound over additive disturbance is set to be $\begin{bmatrix}0.03&0.03&0.03&0\end{bmatrix}$. Given the above settings, SCOTS was able to synthesize controllers in about $12$ minutes ($100$ seconds per robot in average). 

\begin{figure}[t]\label{fig:MA}
	\centering
	\includegraphics[width=0.45\textwidth]{figures/MA.png}
	\caption{Nominal trajectories of unicycles produced by ALTRO}
\end{figure}


%In this example, we are focusing on number of robots. This example consist of 10 identical unicycle model. Each unicycle should start from its own initial point and reach target set which is shown in Fig. \ref{fig:MA} with green points and red points. while it avoids obstacles and collision with other agents. Unicycle model is described in below.
%\begin{equation}\label{eq:unicycle_ss}
%	f^{u}(x(t),u(t))=
%	\begin{bmatrix}
%	\dot{x_1}\\
%	\dot{x_2}\\
%	\dot{x_3}
%	\end{bmatrix}=
%	\begin{bmatrix}
%	u_1cos(x_3)\\
%	u_1sin(x_3)\\
%	u_2
%	\end{bmatrix}.
%\end{equation}
%In Eq. \ref{eq:unicycle_ss} $u_1$ is linear speed and $u_2$ is  angular speed and $x_1$ and $x_2$ represents positions and $x_3$ is the angle. Here goal of the problem is designing a controller which satisfy the specifications which mentioned in sec. \ref{sec:Problem_statement}. For this purpose, According to sec.\ref{sec:nominal trajectory} at first we combine all of unicycle dynamics to create a centralized model of the centralized system. The centralized system has $10\times3$ dimension for state space and $10\times2$ dimension for input space. For collision avoidance and obstacle avoidance we use $\delta_{col}=1.5$ and $\delta_{obs}=2$ and we use euclidean distance as a metric for distance in 2D space. In this step, we solved mentioned problem with ALTRO with sample time 0.05 and 200 points for trajectory. Computation takes 138 seconds. Result of ALTRO is illustrated in fig \ref{fig:MA}. 
%In next step we synthesize close loop controller using ABCD with $\varepsilon=[0.16,0.16,0.16]$.  As we mentioned in Prob. \ref{alg:abcd-with-time-for-tracking} we are going to to design a controller that is able to track the trajectory with $\varepsilon$ distance in every time step which is also robust to the disturbance .\MZ{Here we can split the centralized system to 10 separate system and design a close loop controller for each one of them separately.} In order to do this step, we select $\varepsilon$ in a way that it satisfies $\delta_{col} > 2\sqrt{\varepsilon_x^2+\varepsilon_y^2}$ (since tubes should not intersect with each other). Then we are able to design a controller guaranteeing the specifications for each agent independently using Alg. \ref{alg:abcd-with-time-for-tracking} and SCOTS. We select parameters in this way:\\
%$\widetilde{X}=[-1:11]*[-1:11]*[-\pi,\pi]*[0,1]$\\
%$U=[-3,3]*[-3,3]$\\
%$\eta_{\widetilde{X}}=[0.04,0.04,0.04,0.05]$\\
%$\eta_{U}=[0.6,0.6]$\\
%number of states = about $7*10^7$\\
%reduced number of states = about $7*10^7$\\
%number of inputs =121\\
%
%Abstraction in total for 10 systems takes about 12 minutes(average 100 seconds for each system) and synthesizing controller takes about 2 minutes(near 15 seconds for each system). Controller designed in this way is able to overcome bounded additive disturbance $|w|\leq[0.03,0.03,0.03]$ (for all of unicycles). Nominal controller which is produced by ALTRO with presence of this disturbance will be pushed outside of $\epsilon$ tube and also as it shown in the Fig.\ref{fig:MA}, trajectory 1 and 2 will have collision with this value of disturbance. 




 %Figure~\ref{fig:inv_pend} demonstrates variations of disturbance limit under which SCOTS is able to find a controller for different tube width. %It can be observed that by increasing the tube width, SCOTS is able to synthesize a controller for larger disturbance levels. Figure~\ref{fig:invpend_traj} demonstartes trajectories of the system  \eqref{eq:inv_pend_ssq} governed by controllers synthesized by ALTRO and SCOTS under appliance of constant disturbance vector \MS{$[??,??]$}. As expected, the controller designed by ALTRO is not able to tackle disturbance, while using the controller generated by SCOTS, the trajectory remains within the desired bound.

%\begin{figure}\label{fig:inv_pend}
%	\includegraphics[]{}
%	\caption{Variations of disturbance limit for which SCOTS is able to find a controller with tube size}
%\end{figure}
%\begin{figure}\label{fig:invpend_traj}
%	\includegraphics[]{}
%	\caption{}
%\end{figure}
%\textcolor{red}{
\subsection{Crane and Lifter}
The goal of this example is to study performance of our method for controlling a number of robots with different dynamics. Consider a production line wherein a crane puts some items at certain points and a lifter is in charge of picking these items up. It is assumed that at its lowest height the crane's hook is positioned such that the lifter cannot pass and reach to the object's place. In particular, Figure~\ref{fig:cr_and_lft} shows a scenario wherein the crane has left an object near the left end of the boom and the lifter is initially positioned near the right end of the boom. The goal is to control both the crane and the lifter so that the lifter takes the object without colliding with the crane. 

We model the crane and lifter by tuples $\Sigma^1=(X^1,U^1,W^1,f^c)$ and $\Sigma^2=(X^2,U^2,W^2,f^l)$, respectively.  The crane is modeled as cart-pole system for which the set of state variables $X^1$ consists of $x_1^{(1)}$, $x_2^{(1)}$, $x_3^{(1)}$ and $x_4^{(1)}$ denoting (linear) position and speed of the cart and (angular) position and speed of the pole, respectively. The set of input variables includes $u_1^{(1)}$ which represents linear acceleration of the cart. Finally, crane's dynamics takes the form for an inverted pendulum (see, e.g., \cite{barto1983neuronlike}) and can be written as
\[\begin{bmatrix}
	\dot{x}_1^{(1)}\\
	\dot{x}_2^{(1)}\\
	\dot{x}_3^{(1)}\\
	\dot{x}_4^{(1)}
\end{bmatrix}=\begin{bmatrix}
	x_2^{(1)}\\
	f^c_1(x_3^{(1)},x_4^{(1)},u^{(1)})\\
	x_4^{(1)}\\
	f^c_2(x_3^{(1)},x_4^{(1)},u^{(1)})\\
\end{bmatrix}.\]
\MS{what are $f^c_1$ and $f^c_2$?}
The lifter's dynamics takes the form
\[\begin{bmatrix}
\dot{x}_1^{(2)}\\ \dot{x}^{(2)}_2 \end{bmatrix}=\begin{bmatrix} x^{(2)}_2\\ u^{(2)} \end{bmatrix},
\]
where $x_1^{(2)}$ and $x_2^{(2)}$ denote the lifter's position and speed and $u^{(2)}$ represents lifter's control input (acceleration).

As promised, we use ALTRO in the first step to generate a valid trajectory for the case that $W=\{0\}$. There is no static obstacle defined for this example and for minimum distance between the crane and the lifter we choose $\delta_{col}=0.4$. The sampling time and the horizon length are fixed to $\tau=0.1$ and $N=40$, respectively. Given these settings, ALTRO was capable of generating a valid open-loop trajectory in $10.7$ seconds. Figure \ref{fig:cr_and_lft} demonstrates snapshots of the produced trajectory. Surprisingly enough, applying (constant) additive disturbance $W=\begin{bmatrix}0&0.02&0&0\end{bmatrix}^T$ causes collision between the crane and lifter.

In the next step, we use SCOTS in order to compute a closed-loop controller tolerating additive disturbance with bounds selected according to the vector $\begin{bmatrix}0&0.02&0&0\end{bmatrix}^T$ for different dimensions. We note that the selection of tube sizes need to satisfy the following inequality:
\[ \delta_{col}> \varepsilon_{1}^{(1)} + l\times\varepsilon_{3}^{(1)} + \varepsilon_{1}^{(2)},\]
where $\varepsilon_{1}^{(1)}$, $\varepsilon_{3}^{(1)}$ denote tube sizes over $x_1^{(1)}$ and $x_3^{(1)}$, $l$ denotes the length of the rope between the cart and the hook and $\varepsilon_{1}^{(2)}$ denotes the tube size over $x_1^{(2)}$. 


We choose state and input spaces for the crane to be \MS{$X^{(1)}=??$} and \MS{$U^{(1)}=??$}, respectively. For the lifter, we choose , \MS{$X^{(2)}=??$}, \MS{$U^{(2)}=??$}. Next, we augment both dynamics with time and choose state and input partition sizes \MS{$\eta_{\widetilde{X}^{(1)}}=??$, $\eta_{U^{(1)}}=??$,  $\eta_{\widetilde{X}^{(2)}}=??$ and $\eta_{U^{(2)}}=??$}. These selections yields discretized models with \MS{$|\hat X^{(1)}|=??$, $|\hat U^{(1)}|=??$, $|\hat X^{(2)}|=??$ and $|\hat U^{(2)}|=??$} points for the the crane and lifter, respectively. Next, we limit both of the state spaces into tubes around the (augmented) open-loop trajectories with sizes specified as \MS{$\varepsilon^{(1)}=??$ and $\varepsilon^{(2)}=??$}. This reduces size of state spaces into \MS{$|\hat X^{(1)}|=??$ and $|\hat X^{(2)|}=??$}. Given the above settings, SCOTS was able to synthesize controller for the crane and lifter in about $35$ and $1$ seconds, respectively.
			
\begin{figure}[t]\label{fig:cr_and_lft}
	\centering
	\includegraphics[width=0.45\textwidth]{figures/crane_and_lifter.png}
	\caption{Illustration of the reference trajectories produced by ALTRO for the crane and lifter example}
\end{figure}


%In this example, we show the method can handle multiple heterogeneous dynamical systems. This example is Modelling a special situation in a factory(shown in fig \ref{fig:cr_and_lft}) . Assume there is a crane and and a lifter. Each one of them should reach their destination without having any accident(satisfying specs). The Crane position will change from 0 to 5 and the lifter from 9 to 4.  Here we model the lifter using $\begin{bmatrix} \dot{x_1}\\ \dot{x_2} \end{bmatrix}=\begin{bmatrix} x_2\\ u \end{bmatrix}$ ($x_1$ and $x_2$ are representing position and velocity) and for the crane we use inverted pendulum model with exactly same configuration in \cite{barto1983neuronlike}: \\ 
%\[\begin{bmatrix}
%\dot{x}\\
%\Ddot{x}\\
%\dot{\theta}\\
%\Ddot{\theta}
%\end{bmatrix}=\begin{bmatrix}
%\dot{x_1}\\
%\dot{x_2}\\
%\dot{x_3}\\
%\dot{x_4}
%\end{bmatrix}=\begin{bmatrix}
%x_2\\
%g(x_3,x_4,u)\\
%x_4\\
%f(x_3,x_4,u)\\
%\end{bmatrix}\]
%
%where $x$ represent position of the cart  and $\theta$ is angle of the pole, f and g are non-linear functions. The process of controller design is similar to Sec. \ref{sec:MultiAgent}; At first we find a nominal trajectory with ALTRO with required specifications then we robustifing it using our method with modified ABCD.\\
%In this example we do not have any static obstacle, but for collision avoidance we use euclidean distance as the metric with $\delta=0.4$ for measuring distance between crane and lifter. We consider hook of crane as a point mass with a radius and lifter as several point mass on circumference of it, So we consider minimum distance between the hook and points of the lifter as the distance. ALTRO solve this trajectory generation problem using sampling time =0.1 and number of points=40 in 10.7 seconds.\\
%to guarantee collision avoidance specification we should select tube sizes to satisfy:
%\[ \delta_{col}> \varepsilon_x + l*\varepsilon_\theta + \varepsilon^\prime_x\]
%where $\varepsilon_x$ and $\varepsilon_\theta$ are tube sizes for position and angle in crane model (cart pole) and l is the length of the joint and $\varepsilon^\prime_x$ is tube size for position of the lifter.
%In next phase we use finite abstraction with $\varepsilon_x=9*0.015$ and $\varepsilon_\theta=11*0.015$ and $\varepsilon^\prime_x=9*0.015$. It takes about 1000 seconds for crane (cart-pole system) with (1.15)*($10^{11}$) number of input-states pairs and ($7*10^6$)*(71) reduced number of states and synthesis takes 35 seconds. For lifter abstractions abstraction and synthesis will finish in less than a second.\\
%The nominal controller with disturbances $w_1=[0,0.02,0,0]$ for crane and $w_2=[0,0.1]$ for lifter will fail collision specification.
%
%


%} 



\section{Present Challenges}

\begin{enumerate}
	\item In some examples, it was necessary to consider a larger control input space for the SCOTS part, than what was used in the ALTRO part.
	Otherwise, the SCOTS would not be able to track the nominal trajectory within the given precision using any level of discretization. 
	\item For the ship example, applying the largest disturbance with which SCOTS is able to compute a controller, the trajectory resulted from using the open loop controller does not leave  $\varepsilon-$tube around it; in general magnitude of disturbance for which SCOTS can find a controller is relatively small for most of examples that we have tried.
\end{enumerate}




