% !TEX root = main.tex

\section{Systems and Controllers}

\smallskip
\noindent\textbf{Notation.}\
We denote the set of natural numbers including zero by $\mathbb N$.
We use the notation $\mathbb{R}$ and $\mathbb{R}_{>0}$ to denote respectively the set of real numbers and the set of positive real numbers.
We use superscript $n>0$ with $\mathbb{R}$ and $\mathbb{R}_{>0}$ to denote the Cartesian product of $n$ copies of $\mathbb{R}$ and $\mathbb{R}_{>0}$ respectively.
Given two points $x=(x_1,\ldots, x_n)$ and $y=(y_1,\ldots, y_n)$ in $ \mathbb{R}^n$
%(or $\mathbb{R}^n_{>0}$)
and a relational symbol $\triangleright\in \set{\leq, <, = , >, \geq}$, we write $x\triangleright y$ if $x_i\triangleright y_i$ for every $i\in\set{1,2,\ldots,n}$.
The operator $|\cdot |$ is used to denote both the absolute value of a vector and cardinality of a set, depending on the type of the operand, and the operator $\| \cdot \|$ is used to denote the infinity norm.  

Let $f\colon A\to B$ and $g\colon C\to D$ be two functions.
We define the product function $f\otimes g\colon A\times C\to B\times D $, $f\otimes g \colon (a,c)\mapsto (f(a),g(c))$.
The product is associative and extends to more than two functions in the obvious way.
%
%Throughout, we measure distance between two points, within a space $\mathbb{R}^n$, using the \emph{distance metric} $d\colon \mathbb{R}^p\times \mathbb{R}^p\to \mathbb{R}_{\geq0}$, $d^p\colon \left(u,v\right)\mapsto \|u-v\|$. 
%We extend the definition of $d^p$ to measure distance between a point and a set: for any given $u\in \mathbb{R}^p$ and $V\in \mathbb{R}^p$, $d^p(u,V) \coloneqq \inf_{v\in V}\|u-v\|$.
Given $c\in \mathbb{R}^n$ and $\varepsilon\in \mathbb{R}_{>0}^{n}$, the ball with center $c$ and radius $\varepsilon$ in $\mathbb{R}^n$ is denoted by 
$\ball_\varepsilon(c)\coloneqq \set{x \in \mathbb{R}^n \mid  |x-c|\leq \varepsilon }$.

Let $A$ be a set.
We use the notation $A^\infty$ to denote the set of all finite and infinite sequences formed using the members of $A$. Our control tasks are defined using a subset of Linear Temporal Logic (LTL). 
In particular, we use the \emph{until} operator $\mathcal{U}$ and the \emph{next} operator $\bigcirc$ defined as follows. 
Let $p$ and $q$ be subsets of $\reals^n$ and $\rho=(x_0,x_1,\dots)$ be an infinite sequence of elements from $\reals^n$. 
We write $\rho\models\bigcirc^k p$ if $x_k\in p$. We write $\rho\models p\mathcal{U}q$ if there exists 
$i\in\mathbb{N}$ s.t. $x_i\in q$ and $x_j\in p$ for all $0\leq j<i$. 
For detailed syntax and semantics of LTL, we refer to \cite{baier2008principles} and references therein.

%Let $V$ be a finite set of real-valued variables.
%We use the notation $\sem{V}$ (same symbol but in boldface blue) to denote the set of every possible valuations of the variables in $V$, i.e.\ an element in $\sem{V}$ assigns an unique real number to every variable in $V$.
%It is easy to see that $\sem{V} = \mathbb{R}^{|V|}$.
%Let $U\subset V$ be a set.
%For any $v\in \sem{V}$, we use the notation $v[U]$ to denote the valuation in $\sem{U}$ that assigns the same values to the variables in $U$ as assigned by $v$; $v[U]$ is called the projection of $v$ on to $U$.

%Let $V$ be a finite set of real-valued variables, as above.
%Given any two points $u,v\in \sem{V}$ with $u=(u_1,\ldots,u_{|V|})$ and $v=(v_1,\ldots,v_{|V|})$, we introduce the \emph{element-wise distance function} $d_V\colon \sem{V}\times \sem{V}\to \mathbb{R}^{|V|}_{>0} $, $d_V\colon (x,y) \mapsto (|x_1-y_1|,\ldots, |x_{|V|}-y_{|V|}|)$.
%We sometime exploit the notation and use the same $d_V$ to measure the distance between a set $W\subseteq \sem{V}$ and a point $v\in \sem{V}$, which is defined as follows:
%$d_V(v,W)= (\inf_{w\in W} |v_1-w_1|, \ldots, \inf_{w\in W} |v_{|V|}-w_{|V|}|)$.

% \subsection{System Formalisms}
 
\smallskip
\noindent\textbf{Control Systems.}\
A \emph{control system} $\Sigma = (X, x_{\init}, U, W, f)$
consists of a \emph{state space} $X\subseteq \mathbb{R}^n$,
an \emph{initial state} $x_\init\in X$,
an \emph{input space} $U\subseteq\mathbb{R}^m$, 
a compact \emph{disturbance set} $W\subset \mathbb{R}^n$ containing $0$, and a vector field $f:X\times U\rightarrow X$ modeling the \emph{nominal dynamics} of the system. 
A \emph{trajectory} of $\Sigma$ is a finite or infinite sequence $x_0, x_1, \ldots \in X^\infty$
such that $x_0 = x_{\init}$ and for each $i\geq 0$, there is an $u_i\in U$ and $w_i \in W$ such that
$x_{i+1} = f(x_i, u_i) + w_i$.
A trajectory is \emph{nominal} if $w_i = 0$ for all $i\geq 0$.
Intuitively, a control system represents a (possibly nonlinear) dynamical system with control inputs form $U$
and disturbances from the set $W$.

\smallskip
\noindent\textbf{Product Control Systems.}\
Let $\set{\Sigma^i}_{i\in [1;N]}$, 
$\Sigma^i=(X^i,x_\init^i,U^i,W^i,f^i)$, for $i \in [1;N]$, be a set of $N$ control systems. 
The \emph{product control system} of $\set{\Sigma^i}_{i\in [1;N]}$ is defined as the control system 
$\Sigma^\times = (X^\times, x_\init^\times, U^\times, W^\times, f^\times)$,
where $X^\times \coloneqq X^1 \times \ldots \times X^N$, 
$x_\init^\times \coloneqq (x_\init^1, \ldots, x_\init^N)$,
$U^\times \coloneqq U^1 \times \ldots \times U^N$, 
$W^\times \coloneqq W^1 \times \ldots \times W^N$, 
and $f^\times\coloneqq f^1\otimes \ldots \otimes f^N$.
We use the state projection operator $\proj^{i}\colon X^\times\to X^i$ with $\proj^{i} (x^1,\ldots,x^N) = x^i$. 
For brevity, we write $\set{\Sigma^i}$ instead of $\set{\Sigma^i}_{i\in [1;N]}$, when the range of $i$ is irrelevant or clear from context.

\smallskip
\noindent
\textbf{Sampled-time Systems.}\
We define control systems and their product over discrete time.
Such systems can be obtained as time discretizations of continuous-time nonlinear dynamical systems.
We sketch the connection. 
Consider the tuple $(X, x_\init, U, W, f)$ as above and
suppose that $f: X \times U \rightarrow X$ is such that $f(\cdot, u)$ is locally Lipschitz for all $u\in U$.
Given a time horizon $\tau>0$, an initial state $x_0$, and a constant input $u$, define the \emph{continuous time trajectory} $\zeta_{x_0, u}$
of the system on the time interval $[0, \tau]$ as an absolutely continuous function $\zeta_{x_0,u}: [0, \tau] \rightarrow X$ such that $\zeta_{x_0,u}(0) = x_0$, and
$\zeta_{x_0,u}$ satisfies the differential inclusion $\dot{\zeta}_{x_0,u}(t) \in f(\zeta_{x_0,u}(t), u) + W$ for almost all $t\in [0,\tau]$.
Given $\tau$, $x_0$, and $u$, we define $\Sol(x_0, u, \tau)$ as the set of all $x\in X$ such that there is a continuous time trajectory
$\zeta_{x_0,u}$ with $\zeta(\tau) = x$.
A sequence $x_0,x_1, \ldots$ is a \emph{time-sampled trajectory} if $x_0 = x_\init$ and for each $i\geq 0$, we have
$x_{i+1} \in \Sol(x_i, u_i, \tau)$ for some $u_i \in U$.

Given a continuous-time nonlinear dynamical system with the tuple $(X, x_\init, U, W, f)$ as above and a sampling time $\tau$, there exist techniques to construct
a control system $\Sigma = (X, x_\init, U, W, f_\tau)$ such that every time-sampled trajectory of the continuous-time system is
also a trajectory of $\Sigma$.
We omit the details of the construction; see, e.g., \cite{reissig2016feedback}. 

\begin{comment}
We assume that each control system $\Sigma^i$ in $\set{\Sigma^i}_{i\in [1;N]}$ is equipped with an output map $h^i\colon X^i\rightarrow Y$ where $Y\subseteq\reals^p$ is a shared output space. Recall our motivating example, wherein the crane's state space consists of the cart's position ($z$), the cart's speed ($v_z$), the pendulum's angular position ($\theta$), and angular speed ($v_\theta$). The state space for the vehicle is described with only one state variable $z'$ describing its position. Note that $X^1\subseteq \reals^4$ and $X^2\subseteq \reals$. By defining $h^1(z,v_z,\theta,v_\theta)=\begin{bmatrix}z+l\sin(\theta)&0.6+l\cos(\theta)\end{bmatrix}^T$ and $h^2(z')=\begin{bmatrix}z'&0.4\end{bmatrix}^T$, both state spaces can be mapped to $Y\subseteq \reals^2$ in which we would like to compute their distance.
%For example, in a multi-robot scenario, all of the robots move within a two or three dimensional space independent of their individual dynamics.
We define distance between two control systems by $D\colon X^i\times X^j\rightarrow \reals_{\geq 0}$ with $D(u,v) = \| h^i(u)-h^j(v)\|$. We extend the definition of $D$ to measure distance between a system's state and a set defined over the shared output space: for any given $u\in X^i$ and $V\subseteq Y$, $D(u,V) \coloneqq \inf_{v\in V}\|h^i(u)-v\|$ \Sadegh{This is not right}.
\end{comment}



\begin{comment}
There exists a locally Lipschitz function $f_{con}\colon X\times U\rightarrow X$ and an absolutely continuous function $\beta\colon \reals_{\geq 0} \rightarrow X$ such that for every $x\in X$, constant $u\in U$ and fixed sampling time $\tau>0$, 
\begin{align}
	&\dot \beta = f_{con}(\beta(t),u) \quad \beta(0) =x\nonumber\\
	 &f(x,u) =\beta(\tau).\label{eq:cont_to_disc}
\end{align}
Therefore, we can characterize nominal dynamics $f$ using $f_{con}$ and $\tau$. %For a given time horizon $T$, a (state) \emph{trajectory} of $\Sigma$ is a function $\xi\colon[0;T]\rightarrow X$ such that $\xi(0)=x_0$ and $\xi(\cdot)$ fulfills the following inclusion relation for every $t\in[0;T]$:
For an initial state $x_0$ and a control input $u\in U$, we define
\begin{equation}\label{equ:def_f}
	\Sol_\Sigma(x_0,u)=\set{x\mid x\in f(x_0,u) + W}. 
\end{equation} 
\Sadegh{Trajectory should reflect the discrete nature of the system}

\Sadegh{There are lots of inconsistencies in this definition. I will change this depending on how other things are used in the rest of the paper.}
%We collect all such solutions in the set $\Sol_\Sigma(x_0,u,T)$. 
% such that $f(\cdot,u)$ is locally Lipschitz for all $u\in U$.%, an \emph{output space} $Y\subseteq \mathbb{R}^p$, and
%an \emph{output function} $h\colon X\mapsto Y$. 
%
%We call each element of the set $\sem{X}$ a \emph{state}, and the set $\sem{X} = \mathbb{R}^{|X|}$ the \emph{state space} of $\Sigma$.
%Throughout, we will assume that there is a metric $d\colon \sem{X}\times \sem{X}\to \mathbb{R}$ on the set $\sem{X}$ such that $(\sem{X},d)$ is a metric space. 
%Given a time horizon $T>0$, and a constant input $u\in U$, 
%a (state) \emph{trajectory} of $\Sigma$ 
%on $[0,T]$ is an absolutely continuous function $\xi:[0,T]\rightarrow X$  such that $\xi(0) = x_\init$ and
%$\xi(\cdot)$ fulfills the following differential inclusion for almost every $t\in[0,T]$:
%\begin{equation}\label{equ:def_f}
% \dot{\xi}\in f(\xi(t),u) + W. 
%\end{equation} 
%We collect all such solutions in the set $\Sol_\Sigma(x_0,u,T)$. 
\end{comment}

\begin{comment}

\smallskip
\noindent\textbf{Transition System.}\
A \emph{state-transition system} $S=(Z,z_\init,U,\delta)$ consists of a \emph{state space} $Z$, an \emph{initial state} $z_\init\in Z$, an \emph{input space} $U$, and a \emph{transition function} $\delta:Z\times U \rightarrow 2^Z$. 
A system $S$ is \emph{finite} if $Z$ and $U$ are finite. 
A \emph{trajectory} of $S$ is a maximal sequence of states $\rho = (z_0,z_1,\ldots) \in Z^\infty$ starting at $z_0=z_\init$ and compatible with $\delta$:
for all $1\leq k < |\rho|$ there exists $u\in U$ such that $z_k\in \delta(z_{k-1},u)$ and 
if $|\rho| < \infty$ then $\delta(z_{|\rho|},u)= \emptyset$ for all $u\in U$.
%For $D\subseteq X$, a $D$-trajectory is a trajectory $\xi$ with $\xi(0)\in D$.


\smallskip
\noindent\textbf{Product Transition System.}\
Let $\set{S^i}_{i\in [1;N]}$, $S^i=(Z^i,z_\init^i,U^i,\delta^i)$, be a set of $N$ transition systems.
The \emph{product transition system of } $\set{S^i}_{i\in [1;N]}$ is the transition system $S^\times \coloneqq (Z^\times, z_\init^\times, U^\times, \delta^\times)$ such that $Z^\times \coloneqq Z^1\times \ldots \times Z^N$, $z^\times_\init\coloneqq (z_\init^1,\ldots,z_\init^N)$, $U^\times \coloneqq U^1\times \ldots\times U^N$, and $\delta^\times \coloneqq \delta^1\otimes\ldots \otimes \delta^N$.
From here on, we omit the domain of $i$ and use ``$\set{S^i}$'' and ``$\set{\Sigma^i}$'' respectively for ``$\set{S^i}_{i\in [1;N]}$'' and ``$\set{\Sigma^i}_{i\in [1;N]}$'' to simplify the notation.


\end{comment}

\smallskip\noindent
\textbf{Controllers.} 
%
% We introduce the notations for representing different types of controllers used in this paper. 
An \emph{open-loop controller} for $\Sigma=(X,x_\init,U,W,f)$ over a time interval $[0;T]$ with $T\in\mathbb N$ 
is a function $C\colon [0;T]\to U$. 
The open-loop is obtained when we connect $C$ with $\Sigma$ serially,
 %by taking the input $u_i = C(i)$ for all $i$,
 denoted by $C \triangleright \Sigma$. 
The set of trajectories of the open-loop system $C\triangleright\Sigma$ consists of all finite trajectories
$x_0, x_1, \ldots, x_T$ such that $x_0 = x_\init$ and for all $i \in [0; T-1]$, we have $x_{i+1} = f(x_i, C(i)) + w_i$ for some $w_i \in W$.

A \emph{feedback controller} for $\Sigma$ over a time interval $[0;T]$, $T\in \mathbb N$, 
is a function $C\colon X\times[0;T]\to U$. 
We denote the feedback composition of $\Sigma$ with $C$ as $C \parallel \Sigma$.
The set of trajectories of the closed-loop system $C\parallel\Sigma$ consists of all finite trajectories
$x_0, x_1, \ldots, x_T$ such that $x_0 = x_\init$ and for all $i \in [0; T-1]$, we have $x_{i+1} = f(x_i, C(x_i, i)) + w_i$ for some $w_i \in W$.
For both open-loop and feedback composition, the nominal trajectories are those trajectories for which $w_i = 0$ for all $i\geq 0$.

Now let $\set{\Sigma^i}$ be a set of control systems.
We can define \emph{global} open-loop and feedback controllers by defining the respective controllers on the product system $\Sigma^\times$.
We can also define \emph{local} open-loop and feedback controllers $C^i$ for each $\Sigma^i$.
In this latter case, the set of trajectories of the system $\set{C^i} \triangleright \set{\Sigma^i}$ (respectively, $\set{C^i}\parallel \set{\Sigma^i}$)
are finite sequences 
$x_0^\times, x_1^\times, \ldots, x_T^\times$ such that
$x_0^\times = x_\init^\times$ and for each $j \in [0;T-1]$, we have 
$\proj^{i}(x_{j+1}^\times) = f^i(\proj^i(x_j^\times), C^i(j))+w_{ji}$
(respectively, 
$\proj^{i}(x_{j+1}^\times) = f^i(\proj^i(x_j^\times), C^i(\proj^i(x_j^\times), j))+w_{ji}$)
for some $w_{ji} \in W^i$, for each $i\in [1;N]$.

\smallskip
\noindent
\textbf{Decentralized Controller Synthesis Problem.}
Let $\set{\Sigma^i}$ be a set of control systems.
A (global) \emph{control specification} $\mathcal{L}$ is a set of finite sequences in $(X^\times)^*$.
Intuitively, a control specification specifies a set of ``good behaviors'' of the product system.
The \emph{decentralized open-loop (resp. feedback) controller synthesis problem} asks, given $\set{\Sigma^i}$ and a global control specification $\mathcal{L}$,
to construct a set of local open-loop (resp.\ feedback) controllers $\set{C^i}$ such that
the trajectories of $\set{C^i}\triangleright \set{\Sigma^i}$ (resp.\ $\set{C^i}\parallel \set{\Sigma^i}$) all belong to the global
control specification $\mathcal{L}$.

In particular, we consider \emph{reach-avoid} specifications, written as  $\lnot\avoid\, \mathcal{U}\,\goal$ in linear temporal logic, for two subsets 
$\avoid, \goal \subseteq X^\times$.
A trajectory $x_0, x_1, \ldots$ saisfies the reach avoid specification if there is some $j\in \mathbb{N}$ such that $x_j \in\goal$ and
for all $i \in [0; j-1]$, we have $x_i \not \in\avoid$.

Our notion of product of control systems is a Cartesian product of each component. Thus, we assume that the dynamics of individual control
systems are not coupled.
However, the overall trajectories of the control systems can be coupled through the control objective.
Indeed, our choice of the specification $\Phi$ on the product state space $X^\times$ subsumes many interesting class of control tasks.
For example, we can express situations when the robots have their own local reach-avoid specifications, and they 
need to avoid collision among each other; we consider such a specification in our examples in Sec.~\ref{sec:local reach-avoid}.
In addition, we can also specify more general tasks which cannot be easily decomposed into individual subtasks for the robots.
One example is the \emph{formation control problem}, which we consider in the example in Sec.~\ref{sec:global formation control}, where a set of robots need to 
reach some location while maintaining a given geometric formation. 
Since the formation is defined using the relative positions of the robots, it is not possible to decompose this task into separate local reach-avoid subtasks.

\begin{example}
	Consider our motivating example in which there are two control systems: the crane and the factory vehicle. The detailed models for both of the systems are given in Sec.~\ref{subsec}. The crane starts at $x_\init^1=\begin{bmatrix}0&0&\pi&0\end{bmatrix}^T$, while the vehicle starts at $x_\init^2=8$.
	%The crane is modeled as a control system $\Sigma^1$ with state space $X^1\subseteq\reals^4$, initial state $x_\init^1=\begin{bmatrix}0&0&\pi&0\end{bmatrix}^T$, input space $U^1\subseteq  \reals^2$, a bounded disturbance $W^1\subset\reals^4$ and a fourth-order nominal dynamics $f^c$ (the complete of dynamics can be found in \cite{Barto1983}).
	The goal set for the crane is $\goal^1=\ball_{r_1}(g_1)$, which is a ball centered at $g_1=\begin{bmatrix}5&0&\pi&0\end{bmatrix}^T$ with a desired size $r_1\in\reals^4_{> 0}$. %is a robustness margin parameter. %Similarly, the factory vehicle can be characterize as a second control system $\Sigma^2$ with state space $X^2\subseteq\reals^2$, initial state $x_\init^2=\begin{bmatrix}8&0\end{bmatrix}^T$, input space $U^2\subseteq  \reals^1$, a bounded disturbance $W^2\subset\reals^2$ and a second-order nominal dynamics $f^l$. 
	The goal set for the vehicle is similarly defined as $\goal^2=\ball_{r_2}(g_2)$ with center $g_2=4$ and size $r_2\in\reals_{> 0}$. The goal set over the product space is defined as $\goal=\goal^1\times\goal^2\subseteq \reals^5$. The set $\avoid$ is defined as $\avoid=\set{ x^\times\in X^\times\mid\;D(x^1,x^2)\leq \delta}$ where $x^1$ and $x^2$ denote components of $x^\times$ corresponding to the two systems, $D$ represents geometric distance between the two systems in the two-dimensional space and $\delta$ denotes a minimum safety distance requirement. It can be noticed that despite the fact that dynamics of the two systems are not coupled, to fulfill the reach-avoid specification $\lnot\avoid\, \mathcal{U}\,\goal$, one has to take both of the dynamics into account.
%\RM{give a simple example that shows the dynamics and the objective. In the objective, you can use the distance between two states, etc.}
\qed
\end{example}



%\subsection{Abstraction-Based Controller Synthesis Background}
%Let $\Sigma = (X, U, W, f, \cdot, \cdot)$ be a control system, $\tau>0$ be the sampling time, $\bound$ be a subset of $X$ which imposes a safety specification, $\Xh$ be a given \emph{finite partition} of $\bound$, and $\Uh$ be a \emph{finite subset} of equally spaced (w.r.t.\ infinity norm on $\mathbb{R}^m$) points in the set $U$. A \emph{finite-state abstraction} of $\Sigma$ is a finite state-transition system $(\Xh,\Uh,\fh)$, where $\xh'$ is in $\fh(\xh,u)$ if there is a pair of states $x\in \xh$ and $x'\in \xh'$ such that there is a trajectory $\xi\in \Sol_\Sigma(x,u,\tau)$ with $\xi(\tau)=x'$.
%
%We assume that the size of the partition elements is provided as a vector $\eta_x\in \mathbb{R}^n_{>0}$ which is an input to the abstraction procedure. We also assume that the set $\Uh$ is chosen based on an input-space discretization parameter $\eta_u\in \mathbb{R}^m_{>0}$. 
%
%We assume that there is a black-box procedure named $\findAbs$ which takes as input the description of $\Sigma = (X,U,W,f,\cdot,\cdot)$, a compact subset $\bound$ of the set $X$, a sampling time $\tau>0$, and state space and input space discretization parameters $\eta_x$ and $\eta_u$ respectively, and returns a finite-state abstraction $\widehat{\Sigma} = (\Xh,\Uh,\fh)$ of $\Sigma$ and a feedback refinement relation $Q$ from $\Sigma_\tau$ to $\hat\Sigma$. More details can be found in Sec.~\ref{sec:abcd}.
%here
