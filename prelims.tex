%! TEX ROOT = ./main.tex

\section{Preliminaries}

\smallskip
\noindent\textbf{Notation.}\
We use the notation $\mathbb{R}$ and $\mathbb{R}_{>0}$ to denote the set of real numbers and the set of positive real numbers respectively.
We use superscript $n>0$ with $\mathbb{R}$ and $\mathbb{R}_{>0}$ to denote the cartesian product of $n$ copies of $\mathbb{R}$ and $\mathbb{R}_{>0}$ respectively.
Given two points $x=(x_1,\ldots, x_n)$ and $y=(y_1,\ldots, y_n)$ in $ \mathbb{R}^n$ (or $\mathbb{R}^n_{>0}$) and a relational symbol $\triangleright\in \set{\leq, <, = , >, \geq}$, we write $x\triangleright y$ if for every $i=1,\ldots,n$, $x_i\triangleright y_i$.
The operator $|\cdot |$ is used to denote both the absolute value of a number and cardinality of a set, depending on the type of the operand, and the operator $\| \cdot \|$ is used to denote the infinity norm.  

Let $f\colon A\to B$ and $g\colon C\to D$ be two functions.
We define the product function $f\oplus g\colon A\times C\to B\times D $, $f\oplus g \colon (a,c)\mapsto (f(a),g(c))$.
The product function can be defined for arbitrarily many functions by extending the definition in the obvious way.

Throughout, we measure distance between two points, within a space $\mathbb{R}^n$, using the \emph{element-wise distance metric} $d^n\colon \mathbb{R}^n\times \mathbb{R}^n\to \mathbb{R}^n$, $d^n\colon \left((u_1,\ldots, u_n),(v_1,\ldots,v_n)\right)\mapsto \left(|u_1-v_1|,\ldots,|u_n-v_n|\right)$. 
We extend the definition of $d^n$ to measure distance between a point and a set: for any given $u\in \mathbb{R}^n$ and $V\in \mathbb{R}^n$, $d^n(u,V) \coloneqq \left( \inf_{v\in V}|u_1,v_1|,\ldots, \inf_{v\in V}|u_n-v_n| \right)$.
Given a point $c\in \mathbb{R}^n$ and given the parameter $\epsilon\in \mathbb{R}_{>0}^{n}$, we use the notation $\ball_\epsilon( x)$ to denote the ball in $\mathbb{R}^n$, centered around $c$ and having radius $\epsilon$.
Formally, \[\ball_\varepsilon(c)\coloneqq \set{x
	\in \mathbb{R}^n \mid  d^n(x,c)\leq \varepsilon }.\] 

Let $A$ be a set.
We use the notation $A^\infty$ to denote the set of all finite and infinite sequences formed using the members of $A$.

%Let $V$ be a finite set of real-valued variables.
%We use the notation $\sem{V}$ (same symbol but in boldface blue) to denote the set of every possible valuations of the variables in $V$, i.e.\ an element in $\sem{V}$ assigns an unique real number to every variable in $V$.
%It is easy to see that $\sem{V} = \mathbb{R}^{|V|}$.
%Let $U\subset V$ be a set.
%For any $v\in \sem{V}$, we use the notation $v[U]$ to denote the valuation in $\sem{U}$ that assigns the same values to the variables in $U$ as assigned by $v$; $v[U]$ is called the projection of $v$ on to $U$.

%Let $V$ be a finite set of real-valued variables, as above.
%Given any two points $u,v\in \sem{V}$ with $u=(u_1,\ldots,u_{|V|})$ and $v=(v_1,\ldots,v_{|V|})$, we introduce the \emph{element-wise distance function} $d_V\colon \sem{V}\times \sem{V}\to \mathbb{R}^{|V|}_{>0} $, $d_V\colon (x,y) \mapsto (|x_1-y_1|,\ldots, |x_{|V|}-y_{|V|}|)$.
%We sometime exploit the notation and use the same $d_V$ to measure the distance between a set $W\subseteq \sem{V}$ and a point $v\in \sem{V}$, which is defined as follows:
%$d_V(v,W)= (\inf_{w\in W} |v_1-w_1|, \ldots, \inf_{w\in W} |v_{|V|}-w_{|V|}|)$.

\subsection{System Formalisms}
 
\smallskip
\noindent\textbf{Control System.}\
A \emph{control system} $\Sigma = (X, x_{\init}, U, W, f, Y, h)$
consists of a \emph{state space} $X\subseteq \mathbb{R}^n$,
an \emph{initial state} $x_\init\in X$,
 an \emph{input space} $U\subseteq\mathbb{R}^m$, 
a compact \emph{disturbance set} $W\subset \mathbb{R}^n$, 
a \emph{nominal dynamics} $f:X\times U\rightarrow X$ such that $f(\cdot,u)$ is locally Lipschitz for all $u\in U$,
an \emph{output space} $Y\subseteq \mathbb{R}^p$, and
an \emph{output function} $h\colon X\mapsto Y$. 
%
%We call each element of the set $\sem{X}$ a \emph{state}, and the set $\sem{X} = \mathbb{R}^{|X|}$ the \emph{state space} of $\Sigma$.
%Throughout, we will assume that there is a metric $d\colon \sem{X}\times \sem{X}\to \mathbb{R}$ on the set $\sem{X}$ such that $(\sem{X},d)$ is a metric space. 

Given a time horizon $T>0$, and a constant input $u\in U$, 
a (state) \emph{trajectory} of $\Sigma$ 
on $[0,T]$ is an absolutely continuous function $\xi:[0,T]\rightarrow X$  such that $\xi(0) = x_\init$ and
$\xi(\cdot)$ fulfills the following differential inclusion for almost every $t\in[0,T]$:
\begin{equation}\label{equ:def_f}
 \dot{\xi}\in f(\xi(t),u) + W. 
\end{equation} 
We collect all such solutions in the set $\Sol_\Sigma(x_0,u,\tau)$. 

\smallskip
\noindent\textbf{Transition System.}\
A \emph{state-transition system} $S=(Z,z_\init,U,\delta)$ consists of a \emph{state space} $Z$, an \emph{initial state} $z_\init\in Z$, an \emph{input space} $U$, and a \emph{transition function} $\delta:Z\times U \rightarrow 2^Z$. 
A system $S$ is \emph{finite} if $Z$ and $U$ are finite. 
A \emph{trajectory} of $S$ is a maximal sequence of states $\rho = (z_0,z_1,\ldots) \in Z^\infty$ starting at $z_0=z_\init$ and compatible with $\delta$:
for all $1\leq k < |\rho|$ there exists $u\in U$ such that $z_k\in \delta(z_{k-1},u)$ and 
if $|\rho| < \infty$ then $\delta(z_{|\rho|},u)= \emptyset$ for all $u\in U$.
%For $D\subseteq X$, a $D$-trajectory is a trajectory $\xi$ with $\xi(0)\in D$.

\subsection{System Composition}

\smallskip
\noindent\textbf{Product Control System.}\
Let $\set{\Sigma^i}_{i\in [1;N]}$, $\Sigma^i=(X^i,x_\init^i,U^i,W^i,f^i,Y,h^i)$, be a set of $N$ control systems with the same output space.
The \emph{product control system of } $\set{\Sigma^i}_{i\in [1;N]}$ is the control system $\Sigma^\times = (X^\times, x_\init^\times, U^\times, \set{0^n}, f^\times, Y^\times, h^\times)$ wherein $X^\times\coloneqq X^1\times \ldots \times X^N$ and we assume that $\sum_{i\in [1;N]} n_i=n$, $x_\init^\times\coloneqq (x_\init^1,\ldots,x_\init^N)$, $U^\times\coloneqq U^1\times\dots\times U^N\subset \reals^m$, $f^\times\coloneqq f^{1}\oplus \ldots\oplus f^{N}$, $Y^\times \coloneqq Y^N$, and $h^\times\coloneqq h^1\oplus \ldots \oplus h^N$. 
We use the state projection operator $\proj^{i}\colon X^\times\to X^i$, $\proj^{i}\colon (x^1,\ldots,x^N)\mapsto x^i$.
From here on, we will omit the domain of $i$ and will use ``$\set{\Sigma^i}$'' instead of ``$\set{\Sigma^i}_{i\in [1;N]}$'' for simpler notation.

\smallskip
\noindent\textbf{Product Transition System.}\
Let $\set{S^i}_{i\in [1;N]}$, $S^i=(Z^i,z_\init^i,U^i,\delta^i)$, be a set of $N$ transition systems.
The \emph{product transition system of } $\set{S^i}_{i\in [1;N]}$ is the transition system $S^\times \coloneqq (Z^\times, z^\times, U^\times, \delta^\times)$ such that $Z^\times \coloneqq Z^1\times \ldots \times Z^N$, $z^\times_\init\coloneqq (z_\init^1,\ldots,z_\init^N)$, $U^\times \coloneqq U^1\times \ldots\times U^N$, and $\delta^\times \coloneqq \delta^1\oplus\ldots \oplus \delta^N$.
From here on, we will omit the domain of $i$ and will use ``$\set{S^i}$'' instead of ``$\set{S^i}_{i\in [1;N]}$'' for simpler notation.

\subsection{Controllers}

\smallskip
\noindent\textbf{Sampled-time Abstraction of Control System.}\
Let $\Sigma = (X, x_\init, U, W, f, \cdot, \cdot)$ be a control system and $\tau>0$ be the sampling time.
(The ``$\cdot$''-s in place of the output space and the output function represent their insignificance in the respective sampled-time abstraction; similar convention will be used elsewhere in the paper.)
A \emph{sampled-time abstraction} of $\Sigma$ is the state-transition system $(X,x_\init,U,f_\tau)$, such that $x'$ is in $f_\tau(x,u)$ if and only if there is a trajectory $\xi\in \Sol_\Sigma(x,u,\tau)$ with $\xi(\tau)=x'$.

The \emph{product sampled-time abstraction} of a set of control systems $\set{\Sigma^i} $ is the product transition system of the set of sampled-time abstractions $\set{\Sigma^i_\tau} $.

\smallskip
\noindent\textbf{Open-loop Controller.}\
Let $\Sigma_\tau=(X,x_\init,U,f_\tau)$ be the sampled-time abstraction of the control system $\Sigma=(X,x_\init,U,W,f,\cdot,\cdot)$.
An \emph{open-loop controller of} $\Sigma_\tau$ is a function $C\colon [0;K]\to U$ for some given time horizon $K>0$.
The \emph{open-loop} is obtained when we connect $C$ with $\Sigma_\tau$ serially, and is formalized using the transition system $C \triangleright \Sigma_\tau = (X\times [0;K],x_\init,U,f_\tau^C)$ such that $(x',k+1)$ is in $f_\tau^C((x,k),u)$ if and only if $x'$ is in $f_\tau(x,u)$ and $C(k)=u$.
The \emph{open-loop behavior} $\Beh^\ol(x_\init)$ of $\Sigma_\tau$ under $C$ is the set of all trajectories of the transition system $C \triangleright \Sigma_\tau$.

\smallskip
\noindent\textbf{Feedback Controller.}\
Let $\Sigma_\tau=(X,x_\init,U,f_\tau)$ be the sampled-time abstraction of the control system $\Sigma=(X,x_\init,U,W,f,\cdot,\cdot)$.
A \emph{feedback controller of} $\Sigma_\tau$ is a function $C\colon X\to U$.
The \emph{closed-loop} is obtained when we connect $C$ with $\Sigma_\tau$ in feedback, and is formalized using the transition system $C\parallel\Sigma_\tau = (X,x_\init,U,f_\tau^C)$ such that $x'$ is in $f_\tau^C(x,u)$ if and only if $x'$ is in $f_\tau(x,u)$ and $C(x)=u$.
The feedback controller $C$ when connected to the sampled-time control system $\Sigma_\tau$ can be interpreted as a zero-order-hold controller for the control system $\Sigma$.
The \emph{closed-loop behavior} $\Beh^\cl(x_\init)$ of $\Sigma_\tau$ under $C$ is the set of all trajectories of the transition system $C\parallel\Sigma_\tau$.

\smallskip
\noindent\textbf{Local and Global Controller.}\
Let $\set{\Sigma^i} $ be a set of control systems, $\tau >0$ be a given sampling time, $\set{\Sigma^i} $ be the respective sampled-time abstractions, $\Sigma^\times$ be the product control system of $\set{\Sigma^i} $, and $\Sigma^\times_\tau$ be the product sampled time abstraction of $\set{\Sigma^i} $.
A \emph{global} open-loop (feedback) controller of $\set{\Sigma^i} $ is an open-loop (a feedback) controller of $\Sigma_\tau^\times$.
On the other hand, a set of \emph{local} open-loop (feedback) controllers of $\set{\Sigma^i_\tau} $ is a set $\set{C^i} $ where, for every $i\in [1;N]$, $C^i$ is an open-loop (a feedback) controller of $\Sigma^i_\tau$.

\smallskip
\noindent\textbf{Decentralized Control.}\
Given $\set{\Sigma_\tau^i} $ and a set of open-loop (feedback) controllers $\set{C^i} $, we define the \emph{decentralized open-loop} (\emph{decentralized closed-loop})---denoted by $\set{C^i} \triangleright \set{\Sigma_\tau^i} $ ($\set{C^i} \parallel \set{\Sigma_\tau^i} $)---as the product transition system of $\set{C^i\triangleright \Sigma^i_\tau} $ ($\set{C^i\parallel \Sigma^i_\tau} $).

\subsection{Linear Temporal Logic for Control Specification}

We will use Linear Temporal Logic (LTL) for specifying control tasks; we refer to usual references \cite{baier2008principles} for detailed syntax and semantics of LTL.
Given a pair of predicates $p$ and $q$, as usual, we will use the notation 
$\lnot p$ for \emph{negation} of $p$, 
$p\vee q$ for $p$ \emph{or} $q$,
$p\wedge q$ for $p$ \emph{and} $q$,
$\bigcirc p$ for \emph{next} $p$,
$p\; \mathcal{U}\; q$ for $p$ \emph{until} $q$,
$\square p$ for \emph{always} $p$,
$\lozenge p$ for \emph{eventually} $p$.

Given a control system $\Sigma$, an open-loop (a feedback) controller $C$ of the sampled-time abstraction $\Sigma_\tau$ of $\Sigma$, and an LTL specification $\Phi$ defined over a set of predicates over the state space of $\Sigma$, we will say that $C\triangleright \Sigma_\tau$ ($C\parallel \Sigma_\tau$) realizes $\Phi$ if every trajectory in the set $\Beh^\ol(x_\init)$ ($\Beh^\cl(x_\init)$) satisfies $\Phi$.

Likewise, given $\set{\Sigma_\tau^i} $, a set of open-loop (feedback) controllers $\set{C^i} $, and an LTL specification $\Phi$ defined using a set of predicates over the state space of $\Sigma^\times$, we will say that decentralized open-loop (decentralized closed-loop) realizes $\Phi$ if every trajectory of $\set{C^i}\triangleright \set{\Sigma_\tau^i}$ ($\set{C^i}\parallel \set{\Sigma_\tau^i}$) satisfies $\Phi$.

\subsection{Abstraction-Based Controller Synthesis}

\smallskip
\noindent\textbf{An Additional Safety Restriction.}\
When the state space of the control system is open or unbounded and the trajectories in \eqref{equ:def_f} are allowed to grow in an unbounded fashion, a finite-state abstraction, in a sense that is going to be formalized soon, will be technically infeasible.
So we impose an additional restriction that the system should always remain inside a compact subset of states $\bound$, which is determined according to some known bounds on the state variables.
This is a standard practice in the literature \cite{reissig2016feedback}, and we will implicitly assume that if there is a tracking controller for Prob.~\ref{prob:tracking_with_time}, then there is a tracking controller for Prob.~\ref{prob:tracking_with_time} with the additional safety requirement $\square \bound$.
%
% a given reach-avoid specification \eqref{eq:spec}, then $C$ also realizes the following modified specification:
%\begin{equation}\label{eq:modified spec}
%	x_0 \wedge ([\bigwedge_{1\leq i\leq N}D(x^i,\obs)>\delta_{obs}\; \bigwedge_{1\leq i,j\leq N\;i\neq j}  d(x^i,x^j)>\delta_{\col}]\; \mathcal{U}\;\reach) \wedge \square \bound.
%\end{equation}

\smallskip
\noindent\textbf{Finite-state Abstraction of Control System.}\
Let $\Sigma = (X, U, W, f, \cdot, \cdot)$ be a control system, $\tau>0$ be the sampling time, $\bound$ be a subset of $X$ which imposes a safety specification, $\Xh$ be a given \emph{finite partition} of $\bound$, and $\Uh$ be a \emph{finite subset} of equally spaced (w.r.t.\ infinity norm on $\mathbb{R}^m$) points in the set $U$.
%Note that, since $X$ is unbounded, hence some partition element of $\Xh$ will inevitably be unbounded set.
A \emph{finite-state abstraction} of $\Sigma$ is a finite state-transition system $(\Xh,\Uh,\fh)$, where $\xh'$ is in $\fh(\xh,u)$ if there is a pair of states $x\in \xh$ and $x'\in \xh'$ such that there is a trajectory $\xi\in \Sol_\Sigma(x,u,\tau)$ with $\xi(\tau)=x'$.

In this work, we will use uniformly sized rectangular partition elements to construct the set $\Xh$ from the set $\bound$.
Without going into the detail of the construction, we assume that the size of the partition elements is provided as a vector $\eta_x\in \mathbb{R}^n_{>0}$ which is an input to the abstraction procedure.
Note that, the larger $\eta_x$ is (where comparison is made dimension-wise), the smaller is the state space $\Xh$ resulting in an efficient computation.
On the other hand, the smaller $\eta_x$ is, the better is the precision of the abstraction $\widehat{\Sigma}$ increasing the chance of a successful controller synthesis.

Similar to the state-space partition size $\eta_x$, we also assume that the set $\Uh$ is chosen based on an input-space discretization parameter $\eta_u\in \mathbb{R}^m_{>0}$ that governs the distance between the points in $\Uh$.

\smallskip
\noindent\textbf{Feedback Refinement Relation.}\
Let $\Sigma$ be a control system, $\Sigma_\tau$ be its sampled-time abstraction, and $\widehat{\Sigma}$ be its finite-state abstraction.
A \emph{feedback refinement relation} (FRR) from $\Sigma_\tau$ to $\widehat{\Sigma}$ 
is a relation $Q\subseteq \bound\times \Xh$ s.t.\ 
for all $x\in \bound$ there is some $\xh\in \Xh$ such that $Q(x,\xh)$ and
for all $(x,\xh)\in Q$, we have
\begin{inparaenum}[(i)]
 \item $\Uh_{\widehat{\Sigma}}(\xh)\subseteq U_{\Sigma_\tau}(x)$, and 
 \item $u\in U_{\widehat{\Sigma}}(\xh) \Rightarrow Q(f_\tau(x,u))\subseteq \fh(\xh,u)$,
\end{inparaenum}
where $U_{\Sigma_\tau}(x):=\set{u\in U \mid f_\tau(x,u)\neq \emptyset}$ and $\Uh_{\widehat{\Sigma}}(\xh):=\set{u\in \Uh \mid \fh(\xh,u)\neq \emptyset}$.
We write $\Sigma_\tau \frr{Q} \widehat{\Sigma}$ if $Q$ is an FRR from $\Sigma_\tau$ to $\widehat{\Sigma}$.

\smallskip
\noindent\textbf{Abstraction-based Controller Design.}\
The abstraction-based controller design (ABCD) \cite{reissig2016feedback} is a $3$-step method to find a robust controller for the sampled-time abstraction $\Sigma_\tau$ of the system $\Sigma$:
First, we compute a finite state abstraction $\widehat{\Sigma}$ s.t.\ $\Sigma_\tau \frr{Q} \widehat{\Sigma}$.
Second, we synthesize an abstract controller of the form $\widehat{C}:\Xh\rightarrow \Uh$ for $\widehat{\Sigma}$ using methods from the reactive synthesis literature.
Finally, we obtain the desired controller $C$ as $C:=\widehat{C}\circ Q$.
It is known that this three step process produces a controller $C$ that realizes the given specification on $\Sigma_\tau$ \cite{reissig2016feedback}.

\smallskip
\noindent\textbf{Computation of the Finite-State Abstraction.}\
We assume that there is a black-box procedure named $\findAbs$ which takes as input the description of $\Sigma = (X,U,W,f,\cdot,\cdot)$, a compact subset $\bound$ of the set $X$, a sampling time $\tau>0$, and state space and input space discretization parameters $\eta_x$ and $\eta_u$ respectively, and returns a finite-state abstraction $\widehat{\Sigma} = (\Xh,\Uh,\fh)$ of $\Sigma$ and a relation $Q$ s.t.\ $\Sigma_\tau \frr{Q} \widehat{\Sigma}$.
The actual implementation of $\findAbs$ can be found in \cite{reissig2016feedback}.
 
%\smallskip
%\noindent\textbf{Controller and Closed Loop.}
%Let $\Sigma = (X, U, W, f)$ be a control system and $\tau>0$ be the sampling time.
%A (state-feedback) \emph{controller} for $\Sigma$ is a function $C:X\rightarrow U$.
%The \emph{closed loop}, formed by connecting $C$ in feedback with $\Sigma$, is the control system $\Sigma\parallel C = (X,U,W,f^C)$ whose trajectories fulfill the following differential inclusion for almost every $t\in [0,\tau]$:
%\begin{equation}\label{equ:def_f^C}
% \dot{\xi}\in f(\xi(t),C(\xi(0))) + W. 
%\end{equation} 

%\smallskip
%\noindent\textbf{Systems.} %\label{sec:prelim_abstraction}
%%
%A \emph{system} $S=(X,U,F)$ consists of a state space $X$, an input space $U$, and a transition function $F:X\times U \rightarrow 2^X$. 
%A system $S$ is \emph{finite} if $X$ and $U$ are finite. 
%A trajectory $\xi\in X^\infty$ is a maximal sequence of states compatible with $F$:
%for all $1\leq k < |\xi|$ there exists $u\in U$ s.t.\ $\xi(k)\in F(\xi(k-1),u)$ and 
%if $|\xi| < \infty$ then $F(\xi(|\xi|),u)= \emptyset$ for all $u\in U$.
%For $D\subseteq X$, a $D$-trajectory is a trajectory $\xi$ with $\xi(0)\in D$.
%%The \emph{behavior} $\Beh{S, D}$ of a system $S=(X,U,F)$ w.r.t.\ $D\subseteq X$ 
%%consists of all $D$-trajectories; when $D=X$, we simply write $\Beh{S}$.
%
%\smallskip
%\noindent\textbf{Controllers and Closed Loop Systems.}
%A \emph{controller} $\C=(\Uc,U,\Gc)$ for a system $S=(X,U,F)$
%consists of a controller \emph{domain} $\Uc\subseteq X$, a space of inputs $U$, and a control map 
%$\Gc:\Uc\fun 2^{U}\setminus\set{\emptyset}$ 
%mapping states in its domain to non-empty sets of control inputs.
%The \emph{closed loop system} formed by interconnecting
%$S$ and $\C$ in \emph{feedback} is defined by the system 
% $S^{cl}=(X,U, F^{cl})$ with $F^{cl}:X\times U \rightarrow 2^{X}$ s.t.\ 
% $x'\in F^{cl}(x,u)$ iff $x\in \Uc$ and $u\in\Gc(x)$ and $x'\in F(x,u)$, or $x\notin \Uc$ and $x'\in F(x,u)$.
%
%% 
%% If $\C$ is \emph{static}, the closed loop system simplifies to
%% \begin{equation*}%\label{equ:def_Fcl}
%%  S^{cl}=(X,U,F^{cl})
%%  ~\text{s.t.}~
%%  F^{cl}(x,u)=
%%  \DiCases{F(x,u)}{u=\Gc(x)}{\emptyset}{\text{otherwise}}.
%% \end{equation*}
%
%\smallskip
%\noindent\textbf{Control Problem.}
%We consider specifications given as $\omega$-regular languages 
%whose atomic predicates are interpreted as sets of states. 
%Given a specification $\psi$, a system $S$, and an interpretation of the predicates as sets
%of states of $S$, we write $\semantics{\psi}_S\subseteq \Beh{S}$ 
%for the set of behaviors of $S$ satisfying $\psi$.
%% 
%The pair $\tuple{S,\psi}$ is called a \emph{control problem} on $S$ for $\psi$. 
%% 
%A controller $\C = (\Uc, U, G)$ for $S$ solves $\tuple{S,\psi}$ if
%$\Beh{S^{cl}, \Uc}\subseteq\semantics{\psi}_S$.
%The set of all controllers solving $\tuple{S,\psi}$ is denoted by $\WIN(S, \psi)$.
%
%
%\smallskip
%\noindent\textbf{Feedback Refinement Relations.}
%Let $S_i=(X_i,U_i,F_i)$, $i\in\Set{1,2}$ be two systems with
%$U_2\subseteq U_1$. 
%A \emph{feedback refinement relation} (FRR) from $S_1$ to $S_2$ 
%is a relation $Q\subseteq X_1\times X_2$ s.t.\ 
%for all $x_1\in X_1$ there is some $x_2\in X_2$ such that $Q(x_1,x_2)$ and
%for all $(x_1,x_2)\in Q$, we have
%\begin{inparaenum}[(i)]
% \item $U_{S_2}(x_2)\subseteq U_{S_1}(x_1)$, and 
% \item $u\in U_{S_2}(x_2) \Rightarrow Q(F_1(x_1,u))\subseteq F_2(x_2,u)$
%\end{inparaenum}
%where $U_{S_i}(x):=\SetComp{u\in U_i}{F_i(x,u)\neq \emptyset}$.
%% \begin{equation}\label{equ:Us}
%%  U_{S_i}(x)=\SetComp{u\in U_i}{F_i(x,u)\neq \emptyset}.
%% \end{equation}
%% 
%We write $S_1\frr{Q} S_2$ if $Q$ is an FRR from $S_1$ to $S_2$.
%
%\smallskip
%\noindent\textbf{Abstraction-Based Controller Synthesis (ABCS).}
%Consider two systems $S_1$ and $S_2$, with $S_1\frr{Q} S_2$. 
%Let $\C=(\Uc,U_2,\Gc)$ be a controller for $S_2$. 
%Then, as shown in \cite{ReissigWeberRungger_2017_FRR}, $\C$ 
%can be refined into a controller for $S_1$, defined by 
%$\C\circ Q=(\widetilde{\Uc},U_1,\widetilde{\Gc})$ 
%with $\widetilde{\Uc}= Q^{-1}(\Uc)$, and
%% $\widetilde{\Fc}(z,x)=\SetComp{z'}{\ExQ{\xa\in Q(x)}{z'\in\Fc(z,\xa)}}$, and
%$\widetilde{\Gc}(x_1)=\SetComp{u\in U_1}{\ExQ{x_2\in Q(x_1)}{u\in\Gc(x_2)}}$ for all $x_1\in\widetilde{\Uc}$. 
%% 
%This implies soundness of ABCS. %, as shown in \cite{ReissigWeberRungger_2017_FRR}, Thm. VI.3.
%
%\begin{proposition}[\cite{ReissigWeberRungger_2017_FRR}, Def. VI.2, Thm. VI.3]
%\label{prop:paimpliespt}
%Let $S_1\frr{Q} S_2$ and $\C\in\WIN(S_2,\psi)$ for a specification $\psi$.
%If for all $\xi_1\in \Beh{S_1}$ and $\xi_2\in\Beh{S_2}$ with
%$\dom{\xi_1}=\dom{\xi_2}$ and $(\xi_1(k), \xi_2(k))\in Q$
%for all $k\in \dom{\xi_1}$ holds that  
%$\xi_2\in\semantics{\psi}_{S_2}\Rightarrow\xi_1\in\semantics{\psi}_{S_1}$, 
%then $\C\circ Q \in \WIN(S_1, \psi)$.
%% \begin{equation}
%%  \propImp{
%%  \begin{propConjA}
%%   \xi_2\in\varphi_2[\psi]\\
%%   \AllQ{k\in\N}{(\xi_1(k),\xi_2(k))\in Q}
%%  \end{propConjA}
%% }{\xi_1\in\varphi_1[\psi]}.
%% \end{equation}
%\end{proposition}
%
%
%
%
%\smallskip
%\noindent\textbf{Time-Sampled System.}\
%Given a time sampling parameter $\tau>0$, we define the \emph{time-sampled system} $\St{}(\Sigma,\tau)=(X,U,\Ft{})$ associated with $\Sigma$, 
%where $X$, $U$ are as in $\Sigma$, and the transition function $\Ft{}:X\times U\fun 2^X$ is defined as follows.
%For  all $x\in X$ and $u \in U$, we have $x'\in \Ft{}(x,u)$ iff there exists a solution $\xi\in\ON{Sol}_f(x,\tau,u)$ s.t.\ $\xi(\tau)=x'$.
%
%\smallskip
%\noindent\textbf{Covers.}\ 
%% For vectors $a,b\in(\real{}\cup\Set{\pm\infty})^n$, a \emph{cell} $\hyint{a,b}$ is closed set $\set{x\in\real{n}\mid \forall i\in\set{1,\ldots,n}.a_i\leq x_i \leg b_i}$.
%% To define a \emph{finite} transition system as an abstraction of $\St{}$ 
%A \emph{cover} $\widehat{X}$ of the state space $X$ is a set of
%non-empty, closed hyper-intervals $\hyint{a,b}$ with $a,b\in (\real{}\cup\Set{\pm\infty})^n$ called \emph{cells},
%such that every $x\in X$ belongs to some cell in $\widehat{X}$. 
%% This definition allows for unbounded cells in $\widehat{X}$. 
%% 
%% 
%Given a grid parameter $\eta \in\real{}_{>0}^n$, we say that a point $c\in Y$ is \emph{$\eta$-grid-aligned} if there is $k\in\Z^n$ s.t.\ for each $i\in \set{1,\ldots,n}$,
%$c_i = \alpha_i + k_i\eta_i - \frac{\eta_i}{2}$.
%Further, a cell $\hyint{a,b}$ is \emph{$\eta$-grid-aligned} if there is a $\eta$-grid-aligned point $c$ s.t.\ $a = c - \frac{\eta}{2}$ and
%$b = c + \frac{\eta}{2}$;
%such cells define sets of diameter $\eta$ whose center-points are $\eta$-grid-aligned. 
%% Given a compact set $Y=\hyint{\alpha, \beta}\subseteq X$ with $\alpha,\beta\in\real{n}$ s.t.\ $\beta - \alpha$ is an integer multiple of $\eta$
%
%% To construct a \emph{finite} abstractions of $\St{}$ using such \emph{$\eta$-grid-aligned} cells as abstract states, we consider a compact \emph{region of interest} $Y \subseteq X$ where we assume that $Y = \hyint{\alpha, \beta}$ with $\alpha,\beta\in\real{n}$ s.t.\ $\beta - \alpha$ is an integer multiple of $\eta$. Then we define the \emph{cover} $\widehat{X}$ of $Y$ as the \emph{finite} set of non-overlapping \emph{$\eta$-grid-aligned} cells 
%% % $\hyint{a,b}$ with $a,b\in (\real{}\cup\Set{\pm\infty})^n$, 
%% s.t.\ every $x\in Y$ belongs to exactly one cell $\xa\in\widehat{X}$. 
%% Clearly, if $Y = \hyint{\alpha, \beta}$ with $\alpha,\beta\in\real{n}$ s.t.\ $\beta - \alpha$ is an integer multiple of $\eta$ we have that the set of $\eta$-grid-aligned cells is a \emph{finite cover} for $Y$.
%
%% In the following, we assume the region of interest $Y$ is fixed once and for all.
%% \RM{OMIT?}\footnote{
%% 	$Y$ induces a \emph{finite} \RM{undef:} abstaction by restricting attention to a subset of the state space for control. 
%% 	$Y$ can also be interpreted as a global safety requirement (see Sec.~\ref{sec:problem}).}
%
%\smallskip
%\noindent\textbf{Abstract Systems.}\ 
%An \emph{abstract system} $\Sa{}(\Sigma,\tau,\eta)=(\Xa{},\Ua{},\Fa{})$ for a control system $\Sigma$,
%a time sampling parameter $\tau > 0$, and a grid parameter $\eta \in \mathbb{R}^n_{>0}$
%consists of an abstract state space $\Xa{}$, a finite abstract input space $\Ua\subseteq U$, and an abstract transition function 
%$\Fa{}:\Xa{}\times \Ua{}\rightarrow 2^{\Xa{}}$. To ensure that $\Sa{}$ is finite, we consider a compact \emph{region of interest} $Y=\hyint{\alpha, \beta}\subseteq X$ with $\alpha,\beta\in\real{n}$ s.t.\ $\beta - \alpha$ is an integer multiple of $\eta$. 
%Then we define $\Xa{}=\widehat{Y}\cup\widehat{X}'$ s.t.\ $\widehat{Y}$ is the \emph{finite} set of \emph{$\eta$-grid-aligned} cells covering $Y$ and $\widehat{X}'$ is a finite set of large unbounded cells covering the (unbounded) region $X\setminus Y$. 
%% 
%We define $\Fa{}$ based on the dynamics of $\Sigma$ only within $Y$. That is, for all $\xa\in\Ya{}$, $\xa'\in\Xa{}$, and $u\in\Ua{}$ 
%we require
% \begin{align}\label{eq:next state abs sys 0}
%   \xa'\in \Fa{}(\xa,u) \ \mbox{ if } \ 	\exists\xi\in \cup_{x\in\xa}\ON{Sol}_f(x,\tau,u) \;.\; \xi(\tau) \in \xa'.
% \end{align}
%For all states in  $\xa\in(\Xa{}\setminus\Ya{})$ we have that $\Fa{}(\xa,u)=\emptyset$ for all $u\in\Ua$.
%% $\hyint{a,b}$ with $a,b\in (\real{}\cup\Set{\pm\infty})^n$, 
%% An \emph{abstract system} $\Sa{}(\Sigma,\tau,\eta)=(\Xa{},\Ua{},\Fa{})$ for a control system $\Sigma$,
%% a time sampling parameter $\tau > 0$, and a grid parameter $\eta \in \mathbb{R}^n_{>0}$
%% consists of:
%% % 
%% \begin{inparaenum}[(i)]
%%  \item an abstract state space $\Xa{}$ which forms a finite cover of $X$;
%%  moreover, we assume there is a non-empty subset 
%%  $\Ya{}\subseteq\Xa{}$ which forms a finite cover of $Y$ with $\eta$-grid aligned cells,
%%  \item a finite abstract input space $\Ua\subseteq U$, and
%%  \item an abstract transition function 
%% $\Fa{}:\Xa{}\times \Ua{}\rightarrow 2^{\Xa{}}$ such that 
%% for all $\xa\in(\Xa{}\setminus\Ya{})$ and $u\in\Ua$ we have $\Fa{}(\xa,u)=\emptyset$ 
%% and for all $\xa\in\Ya{}$, $\xa'\in\Xa{}$, and $u\in\Ua{}$ 
%% it holds that
%%  \begin{align}\label{eq:next state abs sys 0}
%%    \xa'\in \Fa{}(\xa,u) \ \mbox{ if } \ 	\exists\xi\in \cup_{x\in\xa}\ON{Sol}_f(x,\tau,u) \;.\; \xi(\tau) \in \xa'.
%%  \end{align}
%% \end{inparaenum}
%We extend $\Fa{}$ to sets of abstract states $\Upsilon\subseteq \Xa{}$ by defining $\Fa{}(\Upsilon,u) := \bigcup_{\xa\in \Upsilon} \Fa{}(\xa,u)$.
%
%While $\Xa{}$ is not a partition of the state space $X$, notice that cells only overlap at the boundary and one can define 
%a deterministic function that resolves the resulting non-determinism by consistently mapping such boundary states to a unique cell covering it. 
%The composition of $\Xa{}$ with this function defines a partition.
%To avoid notational clutter, we shall simply treat $\widehat{X}$ as a partition.
%
%% \smallskip
%% \noindent\textbf{Induced FRR.}\
%% %\label{sec:prelim:FRR}
%% %
%% It was shown in \cite{ReissigWeberRungger_2017_FRR}, Thm. III.5 that the relation
%% $\Qa{}\subseteq X\times \Xa{}$ defined by $(x,\xa)\in\Qa{}$ iff $x\in\xa$
%% is an FRR between $\St{}$ and $\Sa{}$, i.e.,
%% $\St{}\frr{\Qa{}}\Sa{}$.
%% %
%% Hence, we can apply ABCS as described in \REFsec{sec:Prelim_FRR} by computing a 
%% controller for $\Sa{}$ which can then be refined to a controller for $\St{}$ under the pre-conditions of Prop.~\ref{prop:paimpliespt}.% holds. 
%
%\smallskip
%\noindent\textbf{Control Problem.}\
%It was shown in \cite{ReissigWeberRungger_2017_FRR}, Thm. III.5 that the relation
%$\Qa{}\subseteq X\times \Xa{}$, defined by all tuples $(x,\xa)\in\Qa{}$ for which $x\in\xa$,
%is an FRR between $\St{}$ and $\Sa{}$, i.e.,
%$\St{}\frr{\Qa{}}\Sa{}$.
%%
%Hence, we can apply ABCS as described in \REFsec{sec:Prelim_FRR} 
%by computing a controller $C$ for $\Sa{}$ which can then be refined to a controller for $\St{}$ under the pre-conditions of Prop.~\ref{prop:paimpliespt}.% holds. 
%
%More concretely, we consider safety and reachability control problems for the 
%continuous-time system $\Sigma$, which are defined by a set of \emph{static obstacles} $\obstacle \subset X$ which 
%should be avoided and a set of \emph{goal states} $\goal\subseteq X$ which should be reached, respectively.
%Additionally, when constructing $\Sa{}$, we used a compact region of interest $Y\subseteq X$ to ensure \emph{finiteness} of $\Sa{}$ allowing to apply tools from reactive synthesis \cite{MPS95} to compute $C$. 
%This implies that $C$ is only valid within $Y$. 
%We therefore interpret $Y$ as a \emph{global safety requirement} and synthesize a controller which keeps the system within 
%$Y$ while implementing the specification. 
%This interpretation leads to a safety and reach-avoid
%control problem, w.r.t.\ a safe set $\Rset=Y\setminus \obstacle$ and target set $\Tset=\goal \cap\Rset$.
%% formally defined in linear temporal logic (LTL) \cite{baier2008principles} by
%% \begin{subequations}\label{equ:LTLspec}
%% \begin{align}
%% 	\Specs &= \text{always}\; (\Rset{}),\quad\text{and}\label{equ:LTLspec:safe}\\
%% 	\Specr &= \left[\left(\text{always}\; (\Rset )\right)\; \text{until} \; \Tset{}\right].\label{equ:LTLspec:reach} %\wedge \left[\text{eventually}\; T\right]
%% \end{align}
%% \end{subequations}
%As $\Rset{}$ and $\Tset{}$ can be interpreted as predicates over the state space $X$ of $\St{}$, this directly defines the control problems $\tuple{\St{},\Specs}$ and $\tuple{\St{},\Specr}$ via
%% . As $C\in\WIN(\St{},\Spec)$ iff $\Beh{\St{}^{cl}}\subseteq\semantics{\Spec}_{\St{}}$ the latter holds for $\Spec$ in \eqref{equ:LTLspec} iff
%\begin{subequations}\label{equ:Csound}
% \begin{align}
% \semantics{\Specs}_{\St{}}&:=\SetCompX{\xi\in\Beh{\St{}}}{\AllQ{k\in\dom{\xi}}{\xi(k)\in\Rset{}}},~\text{and}\label{equ:Csound:safe}\\
% \semantics{\Specr}_{\St{}}&:=\SetCompX{\xi\in\Beh{\St{}}}{\ExQ*{k\in\dom{\xi}}{
% \begin{propConjA}
%  \xi(k)\in\Tset{}\\
%  \AllQ{k' \leq k}{\xi(k')\in \Rset{}}
% \end{propConjA}
% }}\label{equ:Csound:reach}
%\end{align}
%\end{subequations}
%for safety and reach-avoid control, respectively.
%Intuitively, a controller $C\in\WIN(\St{},\Spec)$ applied to $\Sigma$ is a sample-and-hold controller, which ensures that the specification holds on all closed-loop trajectories \emph{at sampling instances}.%
%\footnote{This implicitly assumes that sampling times and grid sizes are such that no \enquote{holes} occur between consecutive cells visited by a trajectory. This can be formalized by assumptions on the growth rate of $f$ in \eqref{equ:def_f} which is beyond the scope of this paper.}
%
%To compute $C\in\WIN(\St{},\Spec)$ via ABCS as described in \REFsec{sec:Prelim_FRR} we need to ensure that the pre-conditions of Prop.~\ref{prop:paimpliespt} hold. This is achieved by \emph{under-approximating} the safe and target sets by abstract state sets 
%\begin{align}\label{equ:RsetaTseta}
% \Rseta{}=\SetComp{\xa\in\Xa{}}{\xa\subseteq \Rset{}},~\text{and}~
% \Tseta{}=\SetComp{\xa\in\Xa{}}{\xa\subseteq \Tset{}},
%\end{align}
%and defining $\semantics{\Specs}_{\Sa{}}$ and $\semantics{\Specr}_{\Sa{}}$ via \eqref{equ:Csound} by substituting $\St{}$ with $\Sa{}$, $R$ with $\Rseta{}$ and $T$ with $\Tseta{}$. With this, it immediately follows from Prop.~\ref{prop:paimpliespt} that $C\in\WIN\tuple{\Sa{},\Spec}$ can be refined to the controller $C\circ Q\in\WIN{\tuple{\St{},\Spec}}$.
