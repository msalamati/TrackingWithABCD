%! TEX ROOT = ./main.tex

\section{Preliminaries}

\smallskip
\noindent\textbf{Notation.}\
We use the notation $\mathbb{R}$ and $\mathbb{R}_{>0}$ to denote the set of real numbers and the set of positive real numbers respectively.
We use superscript $n>0$ with $\mathbb{R}$ and $\mathbb{R}_{>0}$ to denote the cartesian product of $n$ copies of $\mathbb{R}$ and $\mathbb{R}_{>0}$ respectively.
Given two points $x=(x_1,\ldots, x_n)$ and $y=(y_1,\ldots, y_n)$ in $ \mathbb{R}^n$ (or $\mathbb{R}^n_{>0}$) and a relational symbol $\triangleright\in \set{\leq, <, = , >, \geq}$, we write $x\triangleright y$ if for every $i=1,\ldots,n$, $x_i\triangleright y_i$.
The operator $|\cdot |$ is used to denote both the absolute value of a vector and cardinality of a set, depending on the type of the operand, and the operator $\| \cdot \|$ is used to denote the infinity norm.  

Let $f\colon A\to B$ and $g\colon C\to D$ be two functions.
We define the product function $f\otimes g\colon A\times C\to B\times D $, $f\otimes g \colon (a,c)\mapsto (f(a),g(c))$.
The product function can be defined for arbitrarily many functions by extending the definition in the obvious way.

%Throughout, we measure distance between two points, within a space $\mathbb{R}^n$, using the \emph{distance metric} $d\colon \mathbb{R}^p\times \mathbb{R}^p\to \mathbb{R}_{\geq0}$, $d^p\colon \left(u,v\right)\mapsto \|u-v\|$. 
%We extend the definition of $d^p$ to measure distance between a point and a set: for any given $u\in \mathbb{R}^p$ and $V\in \mathbb{R}^p$, $d^p(u,V) \coloneqq \inf_{v\in V}\|u-v\|$.
Given a point $c\in \mathbb{R}^n$ and given the parameter $\varepsilon\in \mathbb{R}_{>0}^{n}$, we use the notation $\ball_\epsilon( x)$ to denote the ball in $\mathbb{R}^n$, centered around $c$ and having radius $\epsilon$.
Formally, \[\ball_\varepsilon(c)\coloneqq \set{x
	\in \mathbb{R}^n \mid  |x-c|\leq \varepsilon }.\] 

Let $A$ be a set.
We use the notation $A^\infty$ to denote the set of all finite and infinite sequences formed using the members of $A$. Finally, to define our control tasks, we use a subset of Linear Temporal Logic (LTL). In particular, we will use the until ($\mathcal{U}$) and next  ($\bigcirc$) operators. Let $p$ and $q$ be subsets of $\reals^n$, and $\rho=(x_0,x_1,\dots)$ be an infinite sequence of elements from $\reals^n$. We write $\rho\models\bigcirc p$ if $x_1\in p$ and similarly we write $\rho\models p\mathcal{U}q$ if there exists $i\in\mathbb{N}$ s.t. $x_i\in q$ and for all $0\leq j<i$, $x_j\in p$. For detailed syntax and semantics of LTL, we refer to \cite{baier2008principles} and references therein.

%Let $V$ be a finite set of real-valued variables.
%We use the notation $\sem{V}$ (same symbol but in boldface blue) to denote the set of every possible valuations of the variables in $V$, i.e.\ an element in $\sem{V}$ assigns an unique real number to every variable in $V$.
%It is easy to see that $\sem{V} = \mathbb{R}^{|V|}$.
%Let $U\subset V$ be a set.
%For any $v\in \sem{V}$, we use the notation $v[U]$ to denote the valuation in $\sem{U}$ that assigns the same values to the variables in $U$ as assigned by $v$; $v[U]$ is called the projection of $v$ on to $U$.

%Let $V$ be a finite set of real-valued variables, as above.
%Given any two points $u,v\in \sem{V}$ with $u=(u_1,\ldots,u_{|V|})$ and $v=(v_1,\ldots,v_{|V|})$, we introduce the \emph{element-wise distance function} $d_V\colon \sem{V}\times \sem{V}\to \mathbb{R}^{|V|}_{>0} $, $d_V\colon (x,y) \mapsto (|x_1-y_1|,\ldots, |x_{|V|}-y_{|V|}|)$.
%We sometime exploit the notation and use the same $d_V$ to measure the distance between a set $W\subseteq \sem{V}$ and a point $v\in \sem{V}$, which is defined as follows:
%$d_V(v,W)= (\inf_{w\in W} |v_1-w_1|, \ldots, \inf_{w\in W} |v_{|V|}-w_{|V|}|)$.

\subsection{System Formalisms}
 
\smallskip
\noindent\textbf{Control System.}\
A \emph{control system} $\Sigma = (X, x_{\init}, U, W, f)$
consists of a \emph{state space} $X\subseteq \mathbb{R}^n$,
an \emph{initial state} $x_\init\in X$,
 an \emph{input space} $U\subseteq\mathbb{R}^m$, 
a compact \emph{disturbance set} $W\subset \mathbb{R}^n$ and
a \emph{nominal dynamics} $f:X\times U\rightarrow X$ such that $f(\cdot,u)$ is locally Lipschitz for all $u\in U$.%, an \emph{output space} $Y\subseteq \mathbb{R}^p$, and
%an \emph{output function} $h\colon X\mapsto Y$. 
%
%We call each element of the set $\sem{X}$ a \emph{state}, and the set $\sem{X} = \mathbb{R}^{|X|}$ the \emph{state space} of $\Sigma$.
%Throughout, we will assume that there is a metric $d\colon \sem{X}\times \sem{X}\to \mathbb{R}$ on the set $\sem{X}$ such that $(\sem{X},d)$ is a metric space. 

Given a time horizon $T>0$, and a constant input $u\in U$, 
a (state) \emph{trajectory} of $\Sigma$ 
on $[0,T]$ is an absolutely continuous function $\xi:[0,T]\rightarrow X$  such that $\xi(0) = x_\init$ and
$\xi(\cdot)$ fulfills the following differential inclusion for almost every $t\in[0,T]$:
\begin{equation}\label{equ:def_f}
 \dot{\xi}\in f(\xi(t),u) + W. 
\end{equation} 
We collect all such solutions in the set $\Sol_\Sigma(x_0,u,T)$. 

\smallskip
\noindent\textbf{Transition System.}\
A \emph{state-transition system} $S=(Z,z_\init,U,\delta)$ consists of a \emph{state space} $Z$, an \emph{initial state} $z_\init\in Z$, an \emph{input space} $U$, and a \emph{transition function} $\delta:Z\times U \rightarrow 2^Z$. 
A system $S$ is \emph{finite} if $Z$ and $U$ are finite. 
A \emph{trajectory} of $S$ is a maximal sequence of states $\rho = (z_0,z_1,\ldots) \in Z^\infty$ starting at $z_0=z_\init$ and compatible with $\delta$:
for all $1\leq k < |\rho|$ there exists $u\in U$ such that $z_k\in \delta(z_{k-1},u)$ and 
if $|\rho| < \infty$ then $\delta(z_{|\rho|},u)= \emptyset$ for all $u\in U$.
%For $D\subseteq X$, a $D$-trajectory is a trajectory $\xi$ with $\xi(0)\in D$.

%\subsection{System Composition}

\smallskip
\noindent\textbf{Product Control System.}\
Let $\set{\Sigma^i}_{i\in [1;N]}$, $\Sigma^i=(X^i,x_\init^i,U^i,W^i,f^i)$, be a set of $N$ control systems. 
We assume that all of the control systems within $\set{\Sigma^i}_{i\in [1;N]}$ share an output space $Y\subseteq\reals^p$ to which there exists a set of projections of the form $h^i\colon X^i\rightarrow Y$. Remember our motivating example, wherein the crane's state space consists of the cart's position ($z$), the cart's speed ($\dot z$), the pendulum's angular position ($\theta$), and angular speed ($\dot \theta$). The state space for the forklift is described with only one state variable $z'$ describing its position. Note that $X^1\subseteq \reals^4$ and $X^2\subseteq \reals$. Now, defining $h^1=\begin{bmatrix}z+l\sin(\theta)&0.6+l\cos(\theta)\end{bmatrix}^T$ and $h^2=\begin{bmatrix}z'&0.4\end{bmatrix}^T$, both state spaces can be projected to $Y\subseteq \reals^2$ in which we would like to compute distance.
%For example, in a multi-robot scenario, all of the robots move within a two or three dimensional space independent of their individual dynamics.
We define distance between two control systems as $D\colon X^i\times X^j\rightarrow \reals_{\geq 0}$, $D\colon (u,v)\mapsto\| h^i(u)-h^j(v)\|$. We extend the definition of $D$ to measure distance between a system's state and a set defined over the shared output space: for any given $u\in X^i$ and $V\subseteq X^i$, $D(u,V) \coloneqq \inf_{v\in V}\|u-v\|$. The \emph{product control system of } $\set{\Sigma^i}_{i\in [1;N]}$ is the control system $\Sigma^\times = (X^\times, x_\init^\times, U^\times, \set{0^n}, f^\times, D)$ wherein $X^\times\coloneqq X^1\times \ldots \times X^N$ and we assume that $\sum_{i\in [1;N]} n_i=n$, $x_\init^\times\coloneqq (x_\init^1,\ldots,x_\init^N)$, $U^\times\coloneqq U^1\times\dots\times U^N\subset \reals^m$ and $f^\times\coloneqq f^{1}\otimes \ldots\otimes f^{N}$. 
We use the state projection operator $\proj^{i}\colon X^\times\to X^i$, $\proj^{i}\colon (x^1,\ldots,x^N)\mapsto x^i$.
From here on, we will omit the domain of $i$ and will use ``$\set{\Sigma^i}$'' instead of ``$\set{\Sigma^i}_{i\in [1;N]}$'' for simpler notation.

\smallskip
\noindent\textbf{Product Transition System.}\
Let $\set{S^i}_{i\in [1;N]}$, $S^i=(Z^i,z_\init^i,U^i,\delta^i)$, be a set of $N$ transition systems.
The \emph{product transition system of } $\set{S^i}_{i\in [1;N]}$ is the transition system $S^\times \coloneqq (Z^\times, z^\times, U^\times, \delta^\times)$ such that $Z^\times \coloneqq Z^1\times \ldots \times Z^N$, $z^\times_\init\coloneqq (z_\init^1,\ldots,z_\init^N)$, $U^\times \coloneqq U^1\times \ldots\times U^N$, and $\delta^\times \coloneqq \delta^1\otimes\ldots \otimes \delta^N$.
From here on, we will omit the domain of $i$ and will use ``$\set{S^i}$'' instead of ``$\set{S^i}_{i\in [1;N]}$'' for simpler notation.

\subsection{Controllers} Here, we only introduce the notations for representing different types of controllers used in this paper. More detailes can be found in Sec.~\ref{sec:controllers}. Let $\Sigma_\tau=(X,x_\init,U,f_\tau)$ denote sampled time abstraction of a control system $\Sigma$ for a given sample time $\tau>0$. An open-loop controller of $\Sigma_\tau$ is a function $C\colon [0;T]\to U$ for some given time horizon $T>0$. The open-loop is obtained when we connect $C$ with $\Sigma_\tau$ serially (denoted by $C \triangleright \Sigma_\tau$). The \emph{open-loop behavior} $\Beh^\ol(x_\init)$ of $\Sigma_\tau$ under $C$ is the set of all trajectories of the transition system $C \triangleright \Sigma_\tau$. A feedback controller of $\Sigma_\tau$ is a function $C\colon X\to U$. The closed-loop is obtained when we connect $C$ with $\Sigma_\tau$ in feedback (denoted by $C\parallel\Sigma_\tau$). %The \emph{closed-loop behavior} $\Beh^\cl(x_\init)$ of $\Sigma_\tau$ under $C$ is the set of all trajectories of the transition system $C\parallel\Sigma_\tau$. 
Let $\Sigma^\times_\tau$ be the product sampled time abstraction of $\set{\Sigma^i} $.
A global open-loop (feedback) controller of $\set{\Sigma^i} $ is an open-loop (a feedback) controller of $\Sigma_\tau^\times$.
On the other hand, a set of local open-loop (feedback) controllers of $\set{\Sigma^i_\tau} $ is a set $\set{C^i} $ where, for every $i\in [1;N]$, $C^i$ is an open-loop (a feedback) controller of $\Sigma^i_\tau$. Given $\set{\Sigma_\tau^i} $ and a set of open-loop (feedback) controllers $\set{C^i} $, we define the \emph{decentralized open-loop} (\emph{decentralized closed-loop})---denoted by $\set{C^i} \triangleright \set{\Sigma_\tau^i} $ ($\set{C^i} \parallel \set{\Sigma_\tau^i} $)---as the product transition system of $\set{C^i\triangleright \Sigma^i_\tau} $ ($\set{C^i\parallel \Sigma^i_\tau} $).

%here
%\subsection{Abstraction-Based Controller Synthesis Background}
%Let $\Sigma = (X, U, W, f, \cdot, \cdot)$ be a control system, $\tau>0$ be the sampling time, $\bound$ be a subset of $X$ which imposes a safety specification, $\Xh$ be a given \emph{finite partition} of $\bound$, and $\Uh$ be a \emph{finite subset} of equally spaced (w.r.t.\ infinity norm on $\mathbb{R}^m$) points in the set $U$. A \emph{finite-state abstraction} of $\Sigma$ is a finite state-transition system $(\Xh,\Uh,\fh)$, where $\xh'$ is in $\fh(\xh,u)$ if there is a pair of states $x\in \xh$ and $x'\in \xh'$ such that there is a trajectory $\xi\in \Sol_\Sigma(x,u,\tau)$ with $\xi(\tau)=x'$.
%
%We assume that the size of the partition elements is provided as a vector $\eta_x\in \mathbb{R}^n_{>0}$ which is an input to the abstraction procedure. We also assume that the set $\Uh$ is chosen based on an input-space discretization parameter $\eta_u\in \mathbb{R}^m_{>0}$. 
%
%We assume that there is a black-box procedure named $\findAbs$ which takes as input the description of $\Sigma = (X,U,W,f,\cdot,\cdot)$, a compact subset $\bound$ of the set $X$, a sampling time $\tau>0$, and state space and input space discretization parameters $\eta_x$ and $\eta_u$ respectively, and returns a finite-state abstraction $\widehat{\Sigma} = (\Xh,\Uh,\fh)$ of $\Sigma$ and a feedback refinement relation $Q$ from $\Sigma_\tau$ to $\hat\Sigma$. More details can be found in Sec.~\ref{sec:abcd}.
%here