%! TEX ROOT = ./main.tex

\section{Centralized Computation of Nominal Trajectories}\label{sec:nominal trajectory}
We use the notation ``$\sqcup$'' to denote the disjoint union of sets.
We define the \emph{product control system} $\Sigma = (X, U, \set{0^n}, f)$ wherein $X=X^1\sqcup \ldots \sqcup X^N$ and we assume that $|X|=n$, $U=U^1\times\dots\times U^N\subset \reals^m$, and $f=f^{1}\oplus \ldots\oplus f^{N}$. 
Since we use ALTRO to compute the controlled nominal trajectories, and since ALTRO does not support external disturbances, so we ignore the external disturbances in this stage.
Intuitively, the product control system models the joint behavior of the robotic swarm in the absence of external disturbances.

We first lift the given specification $(\set{\init^i}_{i\in [1;N]}, \set{\goal^i}_{i\in [1;N]}, \set{\obs^i}_{i\in [1;N]}, \delta_\col, \delta_\obs)$ to a specification $(\init, \goal, \obs)$ for a \emph{centralized} controller synthesis problem, where $\init$, $\goal$, and $\obs$ are defined in the following:
\begin{itemize}
	\item The initial state $\init\in \sem{X}$ define as $\init \coloneqq (\init^1, \ldots, \init^N) \in \sem{X}$,
	\item the target set $\goal\subseteq \sem{X}$ defined as $\goal \coloneqq \goal^1\times \ldots \times \goal^N \subseteq \sem{X}$, and
	\item the static obstacle $\obs\subseteq \sem{X}$ defined as 
		\begin{equation}
			\obs \coloneqq 
				\Set{ x\in \sem{X} | 
					\begin{array}{c}
						\exists i\in [1;N]\;.\; d^i(x[X^i], \obs^i)\leq \delta_{\obs}\\
						\wedge\\
						 \exists (i,j)\in [1;N]\times [1;N]\;.\; e(x[X^i\cap Z],x[X^j\cap Z])\leq \delta_{\col}
					\end{array}}.
		\end{equation}
\end{itemize}

When the sets $\obs^i$, for every $i\in [1;N]$, are polytopic and the metric $d^i$, for every $i\in [1;N]$, and the metric $e$ are both the infinity norms, then the set $\obs$ is also polytopic.
This is an useful property from the perspective of implementation.

We say that a central controller $C$ of the system $\Sigma_\tau$ realizes the lifted specification $(\init,\goal,\obs)$ if every closed-loop behavior satisfies the following Linear Temporal Logic (LTL) specification:
\begin{equation}\label{eq:spec}
	\init \Rightarrow (\lnot \obs \ \mathcal{U}\ \goal).
\end{equation} 

We use ALTRO to find an open-loop controller for $\Sigma_\tau$ so that the unique open-loop behavior---\emph{unique}, because of absence of disturbance---of $C \triangleright\Sigma_\tau$ satisfies the LTL formula in \eqref{eq:spec}.
Let the unique open-loop behavior of the controlled system $C \triangleright\Sigma_\tau$ be $(x_0,\ldots,x_K)$ for some $K\in \mathbb{N}$.
It can be shown that the set $\set{(x_0[X^i],\ldots,x_K[X^i])}_{i\in [1;N]}$ is a set of valid nominal trajectories for the systems $\set{\Sigma^i_\tau}_{i\in [1;N]}$.

It is worthwhile to mention that, ALTRO requires to provide a fixed target state, whereas we work with a set of target states.
Our heuristic choice of the single target state for ALTRO within the set $\goal$ was driven by the constant $\varepsilon$ introduced in Prob.~\ref{prob:tracking_with_time}:
We choose a state $g\in \goal$ as target so that for every $i\in [1;N]$, $d^i(g[X^i], \sem{X^i}\setminus \goal^i) \geq \varepsilon$.
The existence of such a $g$ for a system $\Sigma^i$ is guaranteed by the choice of $\varepsilon$ in Prob.~\ref{prob:tracking_with_time}.

%The following proposition provides us a way to obtain a set of nominal trajectories $\set{(x_{0_\nom}^i,\ldots,x_{K_\nom}^i)}_{i\in [1;N]}$ from a central controller synthesis problem on the product system:
%
%\begin{proposition}
%	Let $\set{\Sigma^i}_{i\in [1;N]}$ be a set of control systems, $\tau\in \mathbb{R}_0$ be a sampling time, and the tuple $(\set{\init^i}_{i\in [1;N]}, \set{\goal^i}_{i\in [1;N]}, \set{\obs^i}_{i\in [1;N]}, \delta_\col, \delta_\obs)$ be a given specification.
%	
%\end{proposition}

\begin{comment}% Mahmoud's version
We consider a setting wherein each individual robot starts from a (known) initial state $x_0^i\in X^i$ and is asked to go into a \emph{neighborhood} of its corresponding goal state $x_G^i\in X^i$ under the presence of non-zero disturbance. Concatenated initial and goal states of the collective swarm robot system are denoted by $x_0=\begin{bmatrix}{x_0^{1^T},\ldots,x_0^{N^T}}\end{bmatrix}$ and $x_G=\begin{bmatrix}{x_G^{1^T},\ldots,x_G^{N^T}}\end{bmatrix}$.
We denote by $\obs$ the set of polytopic (static) obstacles scattered in $\reals^2$ and by $\reach\subset X$ for the targetted neighborhood of the collective system.%}

%We consider a setting wherein each individual robot starts from a (known) initial state $x_0^i$ and is asked to go into a \emph{neighborhood} of its corresponding goal state $x_G^i$. We denote by $\obs$ the set of polytopic (static) obstacles scattered in $\reals^2$. %For a given control system $\widetilde\Sigma$, the targetted problem is denoted by a tuple $(\reach,obs,x_0)$, where $x_0\in X$ is a designated initial state, and $\reach$ and $obs$ are subsets of the state spaces $X$.
 %Any collision avoidance scheme ensures that minimum distance between robots is always greater than By (static) obstacle $\avoid$ is selected so that it consists a neighbour
For a given sampling time $\tau>0$ and positive constants $\delta_{\col}$ and $\delta_{\obs}$, a feedback controller $C$ \emph{realizes} the desired specification on $\Sigma_\tau$ if the closed-loop behavior $\Beh^\cl(x_0)$ of $C\parallel\Sigma_\tau$ satisfies the following condition: For \emph{every} infinite sequence $(x_0,x_1,\ldots)$ in $\Beh^\cl(x_0)$,
\begin{itemize}
 \item there \emph{exists} a $K$ s.t.\ $x_K\in \reach$ 
 
 \item for \emph{every} $k\leq K$ and every $1\leq i,j \leq N$ we have $d(x^i_k,x^j_k)>\delta_{\col}$ for $i\neq j$, and
 
 \item for \emph{every} $k\leq K$ and every $1\leq i\le N$ we have $D(x^i_k,obs)>\delta_{\obs}$.
 
 \end{itemize}
 %We denote by $\avoid$ the subset of state-time space at which the second condition above does not hold.
 For a given control system $\Sigma$, the targetted problem is denoted by a tuple $(x_0, \reach,\obs, \delta_{\obs},\delta_{\col})$. In linear temporal logic (LTL) notation \cite{baier2008principles}, the above conditions can be succinctly written as:
\begin{equation}\label{eq:spec}
	x_0 \wedge ([\bigwedge_{1\leq i\leq N}D(x^i,\obs)>\delta_{obs}\; \bigwedge_{1\leq i,j\leq N\;i\neq j}  d(x^i,x^j)>\delta_{\col}]\; \mathcal{U}\;\reach) .
\end{equation}
\end{comment}



\section{ABCD Around The Nominal Trajectories}\label{sec:tracking}

In this section, we restrict our attention to an individual system $\Sigma^i$ for $i\in [1;N]$, and synthesize a feedback tracking controller---using ABCD---which can track the respective nominal trajectory $(x_{0_\nom}^i,\ldots,x_{K_\nom}^i)$ up to some given error $\varepsilon$ (Prob.~\ref{prob:tracking_with_time}).
For the rest of the section, we drop the index $i$ for simplicity of notation.

We will use the notation $\mathbf{1}$ to denote the function $\mathbb{R}_{\geq 0}\to \set{1}$.

\subsection{Translation of Prob.~\ref{prob:tracking_with_time} to an LTL Formula}
First, we translate the tracking problem given in Prob.~\ref{prob:tracking_with_time} for the system $\Sigma=(X,U,W,f)$ to an equivalent controller synthesis problem for an LTL specification $\varphi$ for a different system $\widetilde{\Sigma}$.
Essentially, $\widetilde{\Sigma}$ is obtained by appending the state of the system $\Sigma$ with a variable that varies at the same rate as time.
Formally, the system $\widetilde{\Sigma}$ is defined as the tuple $(\widetilde{X}, U, W, \widetilde{f})$ where $\widetilde{X}\coloneqq X\cup \set{s}$, $s$ being a variable taking values in non-negative reals, and $\widetilde{f} \equiv f \oplus \mathbf{1}$.
We extend the metric $d$ defined on the set $\sem{X}$ to the metric $\widetilde{d}$ for the set $\sem{\widetilde{X}}$ as follows:
$\widetilde{d}\colon ((x,s),(x',s')) \mapsto \max\set{d(x,x'), |s-s'|}$, where $|\cdot|$ denotes the absolute value.
Observe that when $d$ is same as the infinity norm, then so is $\widetilde{d}$.

On the other hand, the specification $\varphi$ is obtained as in the following:
First, define the set $P = \Set{(x,k)\in \sem{\widetilde{X}} | \widetilde{d}(x, x_{k_\nom}) > \varepsilon}$.
Owing to the first inequality in \eqref{eq:choice of eps}, $P$ is a connected set.
\KM{Please check!}
We define $\varphi\coloneqq \square P$:
It can be shown that if a feedback controller $C$ of $\Sigma_\tau$ ensures that all the closed loop behaviors of $C\parallel \Sigma_\tau$ satisfies the specification $\varphi$, then $C$ is a tracking controller for the given nominal trajectory.
\KM{TODO: perhaps it's good idea to prove this formally?}

\subsection{Preliminaries of Abstraction-based Controller Design}

\smallskip
\noindent\textbf{An Additional Safety Restriction.}\
When the state space of the control system is open or unbounded and the trajectories in \eqref{equ:def_f} are allowed to grow in an unbounded fashion, a finite-state abstraction, in a sense that is going to be formalized soon, will be technically infeasible.
So we impose an additional restriction that the system should always remain inside a compact subset of states $\bound$, which is determined according to some known bounds on the state variables.
This is a standard practice in the literature \cite{reissig2016feedback}, and we will implicitly assume that if there is a tracking controller for Prob.~\ref{prob:tracking_with_time}, then there is a tracking controller for Prob.~\ref{prob:tracking_with_time} with the additional safety requirement $\square \bound$.
%
% a given reach-avoid specification \eqref{eq:spec}, then $C$ also realizes the following modified specification:
%\begin{equation}\label{eq:modified spec}
%	x_0 \wedge ([\bigwedge_{1\leq i\leq N}D(x^i,\obs)>\delta_{obs}\; \bigwedge_{1\leq i,j\leq N\;i\neq j}  d(x^i,x^j)>\delta_{\col}]\; \mathcal{U}\;\reach) \wedge \square \bound.
%\end{equation}

\smallskip
\noindent\textbf{Finite-state Abstraction of Control System.}\
Let $\Sigma = (X, U, W, f)$ be a control system, $\tau>0$ be the sampling time, $\bound$ be a subset of $X$ which imposes a safety specification, $\Xh$ be a given \emph{finite partition} of $\bound$, and $\Uh$ be a \emph{finite subset} of equally spaced (w.r.t.\ infinity norm on $\mathbb{R}^m$) points in the set $U$.
%Note that, since $X$ is unbounded, hence some partition element of $\Xh$ will inevitably be unbounded set.
A \emph{finite-state abstraction} of $\Sigma$ is a finite transition system $(\Xh,\Uh,\fh)$ s.t.\ $\xh'$ is in $\fh(\xh,u)$ if there is a pair of states $x\in \xh$ and $x'\in \xh'$ s.t.\ there is a trajectory $\xi\in \Sol_\Sigma(x,u,\tau)$ with $\xi(\tau)=x'$.

In this work, we will use uniformly sized rectangular partition elements to construct the set $\Xh$ from the set $\bound$.
Without going into the detail of the construction, we assume that the size of the partition elements is provided as a vector $\eta_x\in \mathbb{R}^n_{>0}$ which is an input to the abstraction procedure.
Note that, the larger $\eta_x$ is (where comparison is made dimension-wise), the smaller is the state space $\Xh$ resulting in an efficient computation.
On the other hand, the smaller $\eta_x$ is, the better is the precision of the abstraction $\widehat{\Sigma}$ increasing the chance of a successful controller synthesis.

Similar to the state-space partition size $\eta_x$, we also assume that the set $\Uh$ is chosen based on an input-space discretization parameter $\eta_u\in \mathbb{R}^m_{>0}$ that governs the distance between the points in $\Uh$.

\smallskip
\noindent\textbf{Feedback Refinement Relation.}\
Let $\Sigma$ be a control system, $\Sigma_\tau$ be its sampled-time abstraction, and $\widehat{\Sigma}$ be its finite-state abstraction.
A \emph{feedback refinement relation} (FRR) from $\Sigma_\tau$ to $\widehat{\Sigma}$ 
is a relation $Q\subseteq \bound\times \Xh$ s.t.\ 
for all $x\in \bound$ there is some $\xh\in \Xh$ such that $Q(x,\xh)$ and
for all $(x,\xh)\in Q$, we have
\begin{inparaenum}[(i)]
 \item $\Uh_{\widehat{\Sigma}}(\xh)\subseteq U_{\Sigma_\tau}(x)$, and 
 \item $u\in U_{\widehat{\Sigma}}(\xh) \Rightarrow Q(f_\tau(x,u))\subseteq \fh(\xh,u)$,
\end{inparaenum}
where $U_{\Sigma_\tau}(x):=\set{u\in U \mid f_\tau(x,u)\neq \emptyset}$ and $\Uh_{\widehat{\Sigma}}(\xh):=\set{u\in \Uh \mid \fh(\xh,u)\neq \emptyset}$.
We write $\Sigma_\tau \frr{Q} \widehat{\Sigma}$ if $Q$ is an FRR from $\Sigma_\tau$ to $\widehat{\Sigma}$.

\smallskip
\noindent\textbf{Abstraction-based Controller Design.}\
The abstraction-based controller design (ABCD) \cite{reissig2016feedback} is a $3$-step method to find a robust controller for the sampled-time abstraction $\Sigma_\tau$ of the system $\Sigma$:
First, we compute a finite state abstraction $\widehat{\Sigma}$ s.t.\ $\Sigma_\tau \frr{Q} \widehat{\Sigma}$.
Second, we synthesize an abstract controller of the form $\widehat{C}:\Xh\rightarrow \Uh$ for $\widehat{\Sigma}$ using methods from the reactive synthesis literature.
Finally, we obtain the desired controller $C$ as $C:=\widehat{C}\circ Q$.
It is known that this three step process produces a controller $C$ that realizes the given specification on $\Sigma_\tau$ \cite{reissig2016feedback}.

\smallskip
\noindent\textbf{Computation of the Finite-State Abstraction.}\
We assume that there is a black-box procedure named $\findAbs$ which takes as input the description of $\Sigma = (X,U,W,f)$, a compact subset $\bound$ of the set $X$, a sampling time $\tau>0$, and state space and input space discretization parameters $\eta_x$ and $\eta_u$ respectively, and returns a finite-state abstraction $\widehat{\Sigma} = (\Xh,\Uh,\fh)$ of $\Sigma$ and a relation $Q$ s.t.\ $\Sigma_\tau \frr{Q} \widehat{\Sigma}$.
The actual implementation of $\findAbs$ can be found in \cite{reissig2016feedback}.

\subsection{Local Abstraction and Synthesis}

%We introduce some notation.
%Given a state $x\in X$ of a control system $\Sigma$, and given a positive real number $\varepsilon$, we use the notation $\ball_\varepsilon(x)$ to denote the ball (with infinity norm) of radius $\varepsilon$ centered around $x$.
%Formally, $\ball_\varepsilon(x):= \set{x'\in X\mid \| x-x' \|\leq \varepsilon}$. 
%In order to be able to break Problem~\ref{prob:reach-avoid} into synthesizing controller $C^{\nom}$ and solving for Problem~\ref{prob:tracking_with_time}, we need to augment the state vector of $\Sigma$ with aditional state variable time.
%Adding time as a state variable, the corresponding augmented control system is represented as $\widetilde{\Sigma}= (\widetilde{X},U,W,\widetilde{f})$ with $\widetilde X= X\times \reals_{\geq 0}$ and $\widetilde f=\begin{bmatrix} f^T& 1\end{bmatrix}^T $.
%%Adding the time variable, the augmented control system is represented as $\widetilde{\Sigma}= (\widetilde{X},U,W,\widetilde{f})$ with $\widetilde X= X\times \reals_{\geq 0}$ and $\widetilde f=\begin{bmatrix} f^T& 1\end{bmatrix}^T $. 
We introduce some notation.
Given a state $\widetilde x=(x,t)$ of a control system $\widetilde \Sigma$, and given a positive real number $\varepsilon$, we use the notation $\ball_\varepsilon(\widetilde x)$ to denote the ball (with infinity norm) of radius $\varepsilon$ centered around $x$ and radius zero for time $t$.
Formally, \[\ball_\varepsilon(\widetilde{x})\coloneqq \set{\widetilde{x}'
	\in \sem{\widetilde{X}} \mid  (x'[X] \in \ball(x)) \land (x'[\set{s}]\equiv x[\set{s}])}.\] 
We use Alg.~ \ref{alg:abcd-with-time-for-tracking} for solving Prob.~\ref{prob:tracking_with_time} using finite abstraction.
% It means we embedding extra state variable, which represents time. Here we use notation $\Psi_\epsilon(\widetilde{x})$ which is very similar to $\ball_\varepsilon(x)$.
%$\Psi_\epsilon$ denotes the ball radius $\varepsilon$ centered around $X$ and radius zero for time $t$.
%Formally:\\ $\Psi_\epsilon(\widetilde{x}=\begin{bmatrix} x \\t \end{bmatrix}):= \set{\widetilde{x}'=\begin{bmatrix}
%	x' \\
%	t'
%	\end{bmatrix}
%	\in \widetilde{X}\mid  \| x-x' \|\leq \varepsilon \land (t=t')}$\\
%we are using Alg \ref{alg:abcd-with-time-for-tracking} for solving Prob \ref{prob:tracking_with_time} using finite abstraction



%Alg.~\ref{alg:abcd-for-tracking} outlines the steps for solving Prob.~\ref{prob:tracking} using finite state abstraction.

\begin{algorithm}
	\caption{ABCD-for-tracking}
	\label{alg:abcd-with-time-for-tracking}
	\begin{algorithmic}[1]
		\Require $\widetilde{\Sigma}=(\widetilde{X},U,W,\widetilde{f})$, $\tau \in \mathbb{R}_{>0}$, $\eta_x\in \mathbb{R}^n_{>0}$, $\eta_u\in \mathbb{R}^m_{>0}$, $((x_0^\nom,t_0),\ldots,(x_K^\nom,t_k))$, $\varepsilon \in \mathbb{R}_{>0}$
		\KM{Did not change the notation in the algo}
		\Ensure Feedback controller $C:X\rightarrow U$ (partial function)
		\State $\bound\gets \emptyset$
		\For{$k$ from $0$ to $K$}
		\State $\bound\gets \Psi_\varepsilon(x_k^\nom,t_k)$
		\EndFor
		\State $(\widehat{\Sigma},Q) \gets \findAbs(\widetilde{\Sigma}, \bound, \tau, \eta_{\widetilde{x}} , \eta_u)$
		\State Synthesize controller $\widehat{C}$ for $\widehat{\Sigma}$ and the reach-avoid specification $(\Psi_\varepsilon(x_K^\nom,t_k), \emptyset, (x_0^\nom,t_0))$
		\State \Return $\widehat{C}\circ Q$
	\end{algorithmic}
\end{algorithm}

%\textcolor{red}{
%\subsection{Local Abstraction considering time and synthesis }
%For this purpose, we build a new control system $\widetilde{\Sigma}= (\widetilde{X},U,W,\widetilde{f})$ from $\Sigma = (X,U,W,f)$ with $\widetilde{X}=\begin{bmatrix} X\\ t\end{bmatrix}$ and $\widetilde{f}=\begin{bmatrix} X\\ 1\end{bmatrix} $. It means we embedding extra state variable, which represents time. Here we use notation $\Psi_\epsilon(\widetilde{x})$ which is very similar to $\ball_\varepsilon(x)$.
%$\Psi_\epsilon$ denotes the ball radius $\varepsilon$ centered around $X$ and radius zero for time $t$.
%Formally:\\ $\Psi_\epsilon(\widetilde{x}=\begin{bmatrix} x \\t \end{bmatrix}):= \set{\widetilde{x}'=\begin{bmatrix}
%x' \\
%t'
%\end{bmatrix}
%\in \widetilde{X}\mid  (x' \in \ball(x)) \land (t=t')}$\\
%we are using Alg \ref{alg:abcd-with-time-for-tracking} for solving Prob \ref{prob:tracking_with_time} using finite abstraction.
%}
%\begin{algorithm}
%	\caption{ABCD-for-tracking}
%	\label{alg:abcd-with-time-for-tracking}
%	\begin{algorithmic}[1]
%		\Require $\widetilde{\Sigma}=(\widetilde{X},U,W,\widetilde{f})$, $\tau \in \mathbb{R}_{>0}$, $\eta_x\in \mathbb{R}^n_{>0}$, $\eta_u\in \mathbb{R}^m_{>0}$, $((x_0^\nom,t_0),\ldots,(x_K^\nom,t_k))$, $\varepsilon \in \mathbb{R}_{>0}$
%		\Ensure Feedback controller $C:X\rightarrow U$ (partial function)
%		\State $\bound\gets \emptyset$
%		\For{$k$ from $0$ to $K$}
%			\State $\bound\gets \Psi_\varepsilon(x_k^\nom,t_k)$
%		\EndFor
%		\State $(\widehat{\Sigma},Q) \gets \findAbs(\widetilde{\Sigma}, \bound, \tau, \eta_{\widetilde{x}} , \eta_u)$
%		\State Synthesize controller $\widehat{C}$ for $\widehat{\Sigma}$ and the reach-avoid specification $(\Psi_\varepsilon(x_K^\nom,t_k), \emptyset, (x_0^\nom,t_0))$
%		\State \Return $\widehat{C}\circ Q$
%	\end{algorithmic}
%\end{algorithm}
